\chapter{Results}

Working on the research so far we have managed to
\begin{itemize}
    \item Get acquainted with the research topic through literature review
    \item Mark and pre-process the original data
    \item Develop the coordinate transformation for conversion of the coordinates of a camera to the view from above
    \item The results of the previous step were used for the solution of the static TSP problem 
    \item For all of the above steps the code has been written, and the model works, giving expected results
\end{itemize}

The next step would be the implementation of the dynamic TSP model and the publication of the results.

\section{Evaluation of Camera Traversal Algorithm }
\subsection{Metric and Experiment formulation}
To evaluate the algorithm, the metric chosen is time required to traverse through all players with stopping time on a player equal to $t_{stop} = 5$. Number of players $n_{players} = 22$ and $n_{iter} = 100$. An experiment will be ran $n_{iter}$ of times, and the T will be the metric for the according algorithm:
$$
T=\frac{1}{n_{iter}}\sum\limits_{i=1}^{n_{iter}} t_{i}
$$

Note that $t_{i}$ is measured in simulation ticks, with conversion formula $25  \cdot t_{i} = 1 \text{ second}$, giving FPS (frames per second) to be 25 ($FPS=25$). Also statistics like standard deviation and interquantile range will be displayed for further analysis. 

Given such framework of evaluation, it is possible to compare quality of a baseline to the quality of the developed KNN-Greedy approach. {\color{purple}(Possibly delete a part about comparing)}

\subsection{Dataset}
To assure that experiments are representative of the real-world scenario, a soccer match simulation from Timur Khaibrakhmanov was used. In the simulation, player movement is modelled with similar to real-world laws. Despite that simulation quite accurately represents football matches, it has a randomization component, that allows to create $n_{iter}=100$ diverse simulations, that represent the soccer dynamics. On those generated simulations, the algorithm is tested. Algorithm only posesses the information prior to during timestamp, thus it simulates real-world unpredictability of players' behaviour.

\subsection{Result of the Experiment}
\begin{table}[h!]
\centering
\caption{Statistics of the KNN Greedy Performance for $N=100$ in frames}
\begin{tabular}{lr}
\toprule
Statistic & Frames for Complete Traversal \\
\midrule
Count     & 100.00000 \\
\textbf{Mean}      & \textbf{408.56000} \\
Standard Deviation (std) & 46.73919 \\
Minimum (min)  & 305.00000 \\
25th Percentile (25\%) & 375.75000 \\
Median (50\%)  & 412.50000 \\
75th Percentile (75\%) & 442.25000 \\
Maximum (max)  & 549.00000 \\
\bottomrule
\end{tabular}
\label{table:algorithm_stats}
\end{table}



\begin{table}[h!]
\centering
\caption{Statistics of the KNN Greedy Performance for $N=100$ in seconds}
\begin{tabular}{lr}
\toprule
Statistic & Seconds for Complete Traversal \\
\midrule
Count     & 100.00000 \\
\textbf{Mean}      & \textbf{16.3424} \\
Standard Deviation (std) & 1.8695676\\
Minimum (min)  & 12.2 \\
25th Percentile (25\%) & 15.03 \\
Median (50\%)  & 16.5 \\
75th Percentile (75\%) & 17.69 \\
Maximum (max)  & 21.96 \\
\bottomrule
\end{tabular}
\label{table:algorithm_stats}
\end{table}




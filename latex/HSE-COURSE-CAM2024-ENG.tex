%% Преамбула TeX-файла

% 1. Стиль и язык
\documentclass[utf8x]{G7-32} % Стиль (по умолчанию будет 14pt)
%\documentclass{book} % Стиль (по умолчанию будет 14pt)
\usepackage[T2A]{fontenc}
\usepackage[russian]{babel}

% Остальные стандартные настройки убраны в preamble.inc.tex.

\usepackage{booktabs}
\usepackage{pdflscape}
\usepackage{longtable}
\usepackage{pdflscape}
\usepackage{lscape}
\usepackage{cancel}
\usepackage{wrapfig}
% Настройки листингов.
\include{listings.inc}
\include{preamble.inc}
\include{macros.inc}
% Полезные макросы листингов.


\title{Курсовая  работа}
\author{}
 

\begin{document}


%Изменим нумерацию на более привычную...
%... и нарушим этим гост.

\renewcommand{\labelenumi}{\arabic{enumi})}
\renewcommand{\labelenumii}{\asbuk{enumii})}


\frontmatter % выключает нумерацию ВСЕГО; здесь начинаются ненумерованные главы: реферат, введение, глоссарий, сокращения и прочее.

% Команды \breakingbeforechapters и \nonbreakingbeforechapters
% управляют разрывом страницы перед главами.
% По-умолчанию страница разрывается.

% \nobreakingbeforechapters
% \breakingbeforechapters


\pagestyle{empty} % нумерация выкл.

\begin{center}
    \textup{\textbf{NATIONAL RESEARCH UNIVERSITY} \\ \textbf{HIGHER SCHOOL OF ECONOMICS}} \\[5mm]

     \textup{Faculty of Computer Science  \\ Bachelor’s Programme “Data Science and Business Analytics”} \\[2mm]

 
       \textup{\large\bfseries
         \\[1mm] Research Team Project Report }\\[5mm] on the topic \\[5mm]
         \textbf{Long-range Camera Guiging for the Person Recognition in Public Spaces.}\\[5mm] (interim, the first stage) 

\end{center}
\vspace{1mm}
\textbf{Fulfilled by the Student:}\\[2mm]
\begin{tabular}{l@{\hskip 1cm}c@{\hskip 1cm}c}
Student of the group \foreignlanguage{russian}{БПАД}212 & & \\
Khaibrakhmanov Timur Radikovich & \rule{3.5cm}{0.15mm}  &  \rule{3.5cm}{0.15mm} \vspace{-2mm} \\
 & \tiny{(signature)}  & \tiny{(date)} \\
\end{tabular} \\
\begin{tabular}{l@{\hskip 1.7cm}c@{\hskip 1cm}c}
Student of the group \foreignlanguage{russian}{БПАД}213 & & \\
Timchenko Daniil Gennadievich & \rule{3.5cm}{0.15mm}  &  \rule{3.5cm}{0.15mm} \vspace{-2mm} \\
 & \tiny{(signature)}  & \tiny{(date)} \\
\end{tabular}\\[3mm]
\textbf{Checked by the Project Supervisor:}\\[2mm]
\begin{tabular}{l@{\hskip 1.5cm}l}
Khelvas Alexander Valerievich\\
COO {JSC COS\&HT} \vspace{1mm}\\
\includegraphics[width=0.32\linewidth]{figures/sign.png} &Feb 14 2024\\
\rule{4cm}{0.15mm} 

 &\rule{4cm}{0.15mm} \vspace{-2mm}\\
{\hskip 1.5cm}\tiny{(signature)} & {\hskip 1.5cm}\tiny{(date)} \\
\end{tabular}


\vspace{\fill}

\begin{center}
Moscow 2024
\end{center}

%\end{titlepage}

% 

\newpage

\textbf{Список исполнителей}
\\[20mm]


  \begin{tabular}{p{150pt}p{100pt}p{100pt}} 
       Руководитель работы & &~ \\ [15pt]
       Исполнитель & &  \\ [25pt]
         & &~ \\ [15pt]

     
       
        
        
    \end{tabular}

\pagestyle{plain}  

% Также можно использовать \Referat, как в оригинале
\begin{abstract}

{\color{red}Общие правила 

1) не используем принудительную разметку страниц 

2) картинки лежат в папках figures и eps (при этом все графики и схемы должны быть в eps)  Шрифт на картинках не должен быть меньше 12. Если на картинке используется текст - должны быть русский и английский варианты. 

3) Все публикации на которые ссылаемся должны быть в bibtex в bib файле

\color{blue} 4) Синим цветом помечены куски которые должны быть переизложены для антиплагиата. Это прямые заимствования}




Авторы  

Хайбрахманов Тимур Радикович (trkhaybrakhmanov@edu.hse.ru)

Тимченко Даниил Геннадьевич (dgtimchenko@edu.hse.ru)

Authors  

Khaibrakhmanov Timur Radikovich

Timchenko Daniil Gennadyevich



Объем отчета - 18 стр.%,  таблиц -1 , приложений - 1.

Ключевые слова:  Алгоритм управления длиннофокусной камерой, Задача коммивояжёра (Travelling Salesman Problem, TSP), Распознавание объектов.
 



\end{abstract}
\pagestyle{plain} % нумерация вкл.
 

\tableofcontents



% %\Defines % Необходимые определения. Вряд ли понадобться

% \chapter{Термины и определения}

% \vspace{20pt}

% \begin{description}

% \item [Искусственный интеллект (далее - ИИ)] - комплекс технологических решений, позволяющий имитировать когнитивные функции человека (включая самообучение и поиск решений без заранее заданного алгоритма) и получать при выполнении конкретных задач результаты, сопоставимые, как минимум, с результатами интеллектуальной деятельности человека. Комплекс технологических решений включает в себя информационно-коммуникационную инфраструктуру, программное обеспечение (в том числе в котором используются методы машинного обучения), процессы и сервисы по обработке данных и поиску решений;
%  \item [g43]
% \end{description}

 
% \Abbreviations %% Список обозначений и сокращений в тексте
% \begin{description}
% \item [ASR] automatic speech recognition (автомаическое распознавание речи)
% \item [CSV]  
%  (CSV от англ. Comma-Separated Values — значения, разделённые запятыми) — текстовый формат, предназначенный для представления табличных данных
% \item [Dynamic HTML]
% набор средств, которые позволяют создавать более интерактивные Web-страницы без увеличения загрузки сервера
% \item [HTML] 
% Язык гипертекстовой разметки документов (от англ. Hypertext Markup Language – “язык гипертекстовой разметки”)
% \item [HTTP] 
% Протокол прикладного уровня для передачи данных, используемый в Web (от англ. HyperText Transfer Protocol - «протокол передачи гипертекста») 
% \item [IP-адрес] 
% Уникальный сетевой адрес узла в компьютерной сети, построенной по протоколу IP
% \item [JavaScript]  Прототипно-ориентированный сценарный язык программирования. Наиболее широкое применение находит в браузерах как язык сценариев для придания интерактивности веб-страницам
% \item [JPEG (JPG)] JPEG - один из популярных графических форматов, применяемый для хранения фотоизображений и подобных им изображений. Файлы, содержащие данные JPEG, обычно имеют расширения .jpg, .jfif, .jpe или .jpeg.
% \item [MS SQL]  Microsoft SQL Server — система управления реляционными базами данных (РСУБД), разработанная корпорацией Microsoft
% \item [PDF]  Portable Document Format (PDF) — межплатформенный формат электронных документов, разработанный фирмой Adobe Systems
% \item [PHP] Cкриптовый язык общего назначения, интенсивно применяемый для разработки веб-приложений.
% \item [PNG]  Растровый формат хранения графической информации, использующий сжатие без потерь качества

% \item [НСИ] 
% Нормативно – справочная информация
% \item [НИР] Научно - исследовательская работа
% \item [АС]  Автоматизированная система
% \item [Интернет]  Информационно-телекоммуникационная сеть Интернет


% \item [Открытые данные] 
% Информация, размещаемая ее обладателями в сети «Интернет» в формате, допускающем автоматизированную обработку без предварительных изменений человеком в целях повторного ее использования

% \item [ПО]  
% Программное обеспечение
% \item[АИС] Автоматизированная информационная система. Но надо протестировать длинные строки в определениях.
% \item[АРМ] Автоматизированная рабочее место
% \item[КВиВ] Подсистема комплексного мониторинга компонент видеомониторинга и видеоанализа 
% \item[МВЯ] Модуль вычислительного ядра  
% \item[МВЯ] Модуль хранения видеоинформации (архив) 
% \item[CPU] Central processing unit 
% \end{description}

% %%% Local Variables:
% %%% mode: latex
% %%% TeX-master: "rpz"
% %%% End:



%\include{12-intro}


\mainmatter % это включает нумерацию глав и секций в документе ниже



% \include{1-Osnovania}  % Постановка задачи

\include{0-Plan-ENG}
\chapter{Description of the problem setting for the course work}
\label{cha:Osnovania-ENG}


\section{Work Plan}

\begin{enumerate}
    \item Familiarize oneself with the literature.
    \item Formulate the goal and objectives of the work.
    \item Transition from a camera view to a top view (Projective Geometry).
    \item Solve the TSP problem for navigating players statically positioned on the field.
    \item Develop a simulation for debugging and testing the long-focus camera control algorithm.
    \begin{enumerate}
        \item Process pre-annotated data to build a heat map of player movement.
        \item Based on heat maps, implement a model of player movement on the field, closely matching their real-game movements.
        \item Develop simulations of camera projection onto the field (3D).
        \item Develop an API to control the camera in the simulation.
        \item Integrate object behavior simulation with camera simulation.
    \end{enumerate}
    \item Develop a metric that fairly evaluates the algorithm's performance.
    \item Develop a camera control algorithm with the best accuracy metrics.
    \begin{enumerate}
        \item Implement field traversal through points most visited by each player without predicting player movement (find the most frequent point for each player).
        \item Represent camera movement as a superposition of camera movement between zones and in a reference system tied to each zone.
        \item Implement player traversal considering predictions of player movement.
    \end{enumerate}
    \item Implement a PTZ camera model considering speed and acceleration constraints imposed on it. In the simplest case, neglect oscillations.
    \item Summarize the results.
\end{enumerate}

Assumptions:

1. The input data is well-labeled and accurately reflects the true locations of objects, with IDs not getting mixed up.

2. One camera has a view of the entire field, while the other is a telephoto camera.

3. If a face is turned within ±45 degrees towards the camera, it is considered possible to photograph (so positions of players are known at a given moment).

4. The height of the players is 170 cm, with the possibility of setting individual values.

5. If the face is captured within 150 pixels under the condition of point 3 and is not obstructed by other players, the player is photographed.

\section{Problem Statement}

Develop a mechanism for controlling a long-focus camera for efficient player detection and tracking within the field of view. A cyclic process is proposed, within which the camera automatically switches between players, adjusting the viewing angle and shooting distance. At the beginning of each iteration, the camera is focused on a new player, and the distance between them and the camera is optimized until the player's silhouette or face occupies a specified percentage of screen space - n\% (with a tolerance). After that, the camera remains fixed on the player for a certain number of frames k before moving to the next player for the next iteration of the process.

When using the term 'efficiently' in this context, it implies that the algorithm should perform optimally on various datasets, such as video recordings of football matches with player labels and their positions in each frame. An algorithm tested on a subset taken from the population of football matches should perform traversal with shooting in the shortest time possible.

\subsection{Coordinate Transformation}
A projectivity from one projective plane to another is called a plane-to-plane projectivity, though it is often simply referred to as a projectivity. It operates on and produces a homogeneous 3-vector, thus represented as a 3-by-3 matrix.

To understand how such a projectivity occurs, consider two images taken from different viewpoints of a plane within a scene, as illustrated in Figure 1. The mapping of points to their corresponding points in image 1 is defined by a projectivity. Similarly, the mapping of points to their corresponding points in image 2 is defined by another projectivity. An essential characteristic of projectivities is that they form a group. Consequently, there exists a projectivity that describes the mapping from the image of the plane in image 1 to the image of the plane in image 2 where.
$$ R = ST^{-1}$$

Given particular coordinates $X,\;Y$ a plane-to-plane projective transformation can be done as following:

$$
\begin{bmatrix}
\tau_{i}X' \\
\tau_{i}Y' \\
\tau_{i}
\end{bmatrix} = 
\underbrace{ \begin{bmatrix}
a_{1} & a_{2} & b_{1} \\
a_{3} & a_{4} & b_{2} \\
c_{1} & c_{2} & 1
\end{bmatrix} }_{ M } \begin{bmatrix}
X \\
Y \\
1
\end{bmatrix}
$$

Where $a_i$ are elements of a scaling/rotation matrix, $\begin{bmatrix}
    b_2 & b_1
\end{bmatrix}^T$ is a translation vector and $\begin{bmatrix}
    c_1 & c_2
\end{bmatrix}$ is a projection vector.

To find true new coordinates $X', Y'$ resulting vector has to be divided by $\tau_i$ that is the scaling factor. 

\subsubsection{Code implementation}

Given source image field corner coordinates in a list corner\_src\_points, a projective transformation matrix can be calculated. Function cv2.getPerspectiveTransform takes 2 arguments: source (4 coordinates (x,y), resembling corners of the input quadrilateral) and destination (4 coordinates (x,y), resembling corners of the output quadrilateral). On output the projective transformation matrix $M \in \mathbb{R}^{3 \times 3}$ described above is obtained.

\lstinputlisting[language=Python]{listings/projective_matrix.py}
\lstinputlisting[language=Python]{listings/coordinates_trans.py}
% add a link to library
Now by using this function and warpPerspective function from \textcolor{purple}{opencv} library, transformation can be done:
\begin{lstlisting}
corrected_image = cv2.warpPerspective(image, M, (width, height), borderValue=(255,255,255))
transform_coordinates(file_name="unscaled_track_df_new_coords.csv")
\end{lstlisting}

\section{Structure of the data}

Given a football field on which there are players and a ball, specified by coordinates $\vec X$, as functions of time $t$, in the reference system associated with the field. So we have an input array $X^{nm}_i$, where

$n$ - player number;

$m$ - describes one of the coordinates of the player’s position;

$i$ - describes a moment in time.

It is necessary to build a high-resolution camera axis control function (reserve $k$ for the camera number) with the given characteristics:

$F=(F_{min},F_{max})$ - focal length range;

$U=(u_1,u_2)$ - camera matrix size in pixels;

$p$ - matrix pixel size in meters (real world size of an image sensor's pixel);

$\Omega=(\omega_1, \omega_2)$ - maximum angular velocity in elevation and azimuth;

$\frac{dF}{dt}_{max}$ - maximum rate of change of focal length over time.

Camera coordinates in the reference system associated with the far left corner of the field

$$W=\{w_1,w_2,w_3\}$$

%Перенести в список терминов и определений

% \subsection{Фокусное расстояние (Focal Length)}
% Focal length is a distance between "nodal point" (that is where light converges in a lens) and a camera sensor\cite{FocalLength}. Cameras have a base focal length (max), but some cameras provide with a possibility to vary focal length by increasing/decreasing length of an objective (объектив). Thus a range of focal length ($F=(F_{min},F_{max})$) is of interest, as applications imply usages with long focus lenses.

%  \begin{figure}[!htbp]
%      \centering
%      \includegraphics[width=0.8\linewidth]{figures/focal.jpg}
%      \caption{Focal Length}
%      \label{fig:enter-label}
%  \end{figure}

% \subsection{Матрица камеры (Image sensor)}
%  An image sensor refers to the electronic component in a digital camera that captures and converts light into digital signals, ultimately creating a digital image. The image sensor plays a crucial role in digital photography by replacing the traditional film used in film cameras. $U=(u_1,u_2)$ - sensor size of a camera in pixels represents number of pixels along $x$ and $y$ axes respectively, total image might have upmost $u_1*u_2$ pixels, given that photo is RGB, it can be calculated, that on an 3-dimensional tensor with shape $(u_1, u_2, 3)$ the whole image can be stored, and on 4-dimensional tensor with shape $(u_1, u_2, 3, \textbf{frames})$ a whole video may be stored frome such camera without audio-stream, where frames - is the amount of frames taken from that camera consequently.



% \subsection{Угловая Скорость (Angular Velocity)}
% An angular velocity is the speed of rotation for an object that can be stated as ${\displaystyle \omega ={\frac {d\varphi }{dt}}}$.

% \subsection{Угол Места; Элевация (Elevation)}
% Vertical angle of an observed object over true horizon. Elevation combined with azimuth are used for obtaining the direction to an object. \href{https://ru.wikipedia.org/wiki/%D0%A3%D0%B3%D0%BE%D0%BB_%D0%BC%D0%B5%D1%81%D1%82%D0%B0}{Elevation}

% \subsection{Азимут (Azimuth)}
% Horizontal angle evaluated from predefined direction (for example north) and direction of an observed object.

\section{The simplest task of controlling a high-resolution camera}

It is necessary to propose an algorithm for bypassing all players on the field, starting from the center of the field.

As a result we should get:

$\vec \psi(t)$ is a vector describing the elevation angle and azimuth of the camera sighting as a function of time.

At the same time, we must ensure that the player’s image is obtained in the camera’s field of view during the time $\Delta T$ corresponding to $R$ frames.

We consider the movement of the players to be a priori unknown.

1) The first step is to bypass stationary players

2) Second step - bypassing moving players

\subsection{Bypassing stationary players}

Given a weighted graph $G = (V, E)$, in our case complete, since from any vertex it can be rational to move to another, in the general case, where

$V$ - number of objects recognized in the frame, at this stage football players, $V = {1, 2, 3, ..., k}$

$E$ - edges of this graph;

$d_{ij}(i, j \in V, i \neq j)$ - the distance between two vertices $i, j$, and $d_{ij} > 0$, $d_{ij} \neq \infty$ and $d_{ij} = d_{ji}$.

We need to find a Hamiltonian cycle (closed path) such that
\begin{align}
     minD = \sum_{i,j \in V} d_{ij} * x_{ij},
\end{align}
Where
\begin{equation}
     x_{ij} =
     \begin{cases}
         1 & e_{ij} \in \text{optimal path}\\
         0 & e_{ij} \notin \text{optimal path}
     \end{cases}
\end{equation}
That is, $x_{ij}$ is a logical variable that turns to 1 if the edge $e_{ij}$ satisfies the condition of belonging to the optimal path, and 0 otherwise. The start and end of the round generally take place in the center of the football field. The result of the algorithm should be a path containing all the vertices of the graph and satisfying the conditions specified above.

\section{Approximations and limitations}
 
Moving the camera angle up/down left/right and focusing are independent of each other and can be done in parallel. (The metric being optimized depends on this)

 
We plan to have a different number of facilities and much greater than the number of football players. (The choice of algorithm depends on this - since the problem is NP hard, for a small amount it can be solved head-on - the 22 hypothesis is more optimal)

The camera should return to the starting point (center by default).

 
% It is impossible to assume that the running speed of a football player is negligible relative to the speed of the camera.


\include{2-Formal-ENG}     
 

\chapter{Literature Review}

{\color{red} In this section, we compile references and literature review on the thesis topic. The text is just a machine translation of annotations - it needs to be fully creatively rethought. }

In the article \cite{Tinos2015}, the dynamic TSP with variable weights is explored. The impact of these changes on the problem's fitness landscapes is examined.

The paper addresses key questions about the dynamic TSP, such as "how many solutions are affected by a change?" and "how does the severity of the problem influence the optimal solutions?"

The study includes simulations of the dynamic TSP with weight changes, revealing that the new optimal solutions are usually close to the previous ones.

In \cite{10.1007/11903697_31}, a robust algorithm for solving DTSP is introduced. Experimental results demonstrate that this algorithm is highly effective, producing high-quality solutions in very short time steps.

In the article \cite{Gharehchopogh2012} ... 

\section{Reviewed Literature List}
\begin{enumerate}
    \item \href{https://www.youtube.com/watch?v=zjMuIxRvygQ&pp=ygULcXVhdGVybmlvbnM%3D}{Quaternions and 3d rotation, explained interactively}
    \item \href{https://www.youtube.com/watch?v=d4EgbgTm0Bg&pp=ygULcXVhdGVybmlvbnM%3D}{Visualizing quaternions (4d numbers) with stereographic projection}
    \item \href{https://distill.pub/2021/gnn-intro/}{A Gentle Introduction to Graph Neural Networks}
    \item \href{https://pubmed.ncbi.nlm.nih.gov/34520362/}{Solving Dynamic Traveling Salesman Problems With Deep Reinforcement Learning}
    \item \href{}{M. Mavrovouniotis and S. Yang, “Ant colony optimization algorithms with immigrants schemes for the dynamic travelling salesman problem,” in Evolutionary Computation for Dynamic Optimization Problems. Berlin, Germany: Springer, 2013, pp. 317–341} (Meta-Heuristics)
    \item \href{}{S. Jiang and S. Yang, “A benchmark generator for dynamic multi-objective optimization problems,” in Proc. 14th UK Workshop Comput. Intell. (UKCI), Sep. 2014, pp. 1–8.} (Meta-Heuristics)
    \item \href{}{C. Groba, A. Sartal, and X. H. Vázquez, “Solving the dynamic traveling salesman problem using a genetic algorithm with trajectory prediction: An application to fish aggregating devices,” Comput. Oper. Res., vol. 56, pp. 22–32, Apr. 2015.} (Meta-Heuristics)
    \item \href{}{J.-F. Cordeau, G. Ghiani, and E. Guerriero, “Analysis and branch-and-cut algorithm for the time-dependent travelling salesman problem,” Transp. Sci., vol. 48, no. 1, pp. 46–58, Feb. 2014} (Linearization of TDTSP)
    \item \href{}{A. Montero, I. Méndez-Díaz, and J. J. Miranda-Bront, “An integer programming approach for the time-dependent traveling salesman problem with time Windows,” Comput. Oper. Res., vol. 88, pp. 280–289, Dec. 2017.} (Linearization of TDTSP)
    \item \href{https://cnrrobertson.github.io/other/mlseminar/fall_2021/Stochastic%20Temporal%20Networks%20-%20Binan%20Gu.pdf}{Stochastic Temporal Networks}
    \item \href{https://towardsdatascience.com/temporal-graph-learning-in-2024-feaa9371b8e2}{Temporal Graph Learning in 2024}
    \item \href{https://link.springer.com/book/10.1007/b101971}{The Traveling Salesman Problem and Its Variations}
    \item \href{https://www.sciencedirect.com/science/article/pii/S1572528608000339}{The time-dependent traveling salesman problem and single machine scheduling problems with sequence dependent setup times Louis-Philippe Bigras, Michel Gamache, Gilles Savard} (Very similar problem statement)
    \item \href{https://arxiv.org/abs/1803.08475}{W. Kool, H. Van Hoof, and M. Welling, “Attention, learn to solve routing problems!,” in Proc. Int. Conf. Learn. Represent., 2019.}
    \item \href{https://medium.com/stanford-cs224w/tackling-the-traveling-salesman-problem-with-graph-neural-networks-b86ef4300c6e}{Tackling the Traveling Salesman Problem with Graph Neural Networks}
\end{enumerate}

\section{Description of Relevant Ideas}

\subsection{Temporal Graphs}

$\bullet$ Each edge $e = (u, v, t) \in E$ is a temporal edge from a vertex $u$ to another vertex $v$ at time $t$. For any two temporal ages $(u,v,t_1)$ and $(u,v,t_2)$ $t_1 \neq t_2$.

$\bullet$ Each vertex $v \in V$ is active when there is a temporal edge that starts or ends at $v$.

$\bullet$ $d(u, v)$: the number of temporal edges from $u$ to $v$ in $G$.

$\bullet$ $E(u, v)$: the set of temporal edges from $u$ to $v$ in $G$, i.e., $E(u,v)=\{(u,v,t_1),(u,v,t_2), ..., (u,v,t_{d(u,v)})\}$.

$\bullet$ $N_{out}(v)$ or $N_{in}(v)$: the set of out-neighbors or in-neighbors of $v$ in $G$, i.e., $N_{out}(v) = \{u : (v, u, t) ∈ E\}$ and $N_{in}(v) = \{u : (u, v, t) ∈ E\}$.

$\bullet$ $d_{out}(v)$ or $d_{in}(v)$: the temporal out-degree or in-degree of $v$ in $G$, defined as $d_{out}(v) = \sum_{u \in N_{out(v}} d(v,u)$ and $d_{in}(v) = \sum_{u \in N_{in(v}} d(u,v)$.

\subsection{Subtypes of TSP}
Based on the book "The Traveling Salesman Problem and Its Variations," two variations of TSP stand out as most relevant to our problem:

{\color{blue}
Moving Target TSP: A set $X = \{x_1, x_2, \ldots, x_n\}$ of $n$ objects placed at points $\{p_1, p_2, \ldots, p_n\}$. Each object $x_i$ is moving from $p_i \in \mathbb{R}^2$ at velocity $v_i$. A pursuer starting at the origin moving with speed $v$ wants to intercept all points $x_1, x_2, \ldots, x_n$ in the fastest possible time. This problem is related to the time-dependent TSP.

Time-Dependent TSP: For each arc $(i, j)$ of $G$, $n$ different costs $c_{ij}^t = 1, 2, \ldots, n$ are given. The cost $c_{ij}^t$ corresponds to the 'cost' of going from city $i$ to city $j$ in time period $t$. The objective is to find a tour $(\tau(1), \tau(2), \ldots, \tau(n), \tau(1))$, where $\tau(1) = 1$ corresponds to the home location which is in time period zero, in $G$ such that $\sum_{i=1}^{n} c_{\tau(i) \tau(i+1)}^{t_i}$ is minimized. The index $n + 1$ is equivalent to 1. For all $(i, j)$, if $c_{ij}^t = c_{ji}^t = \ldots = c_{ij}^{t'}$, then the time-dependent TSP reduces to the traveling salesman problem.
}

In our case, it seems that the problem encompasses both subtypes.

\subsection{Deep Reinforcement Learning}
In 2023, an article \cite{RL} was published that formulates a problem similar to the one considered in this work. It turned out that our problem is more related to Time-Dependent TSP (TDTSP) rather than Dynamic TSP (DTSP), as initially assumed. 
The authors used a Deep Reinforcement Learning approach to solve this problem, introducing an additional complication: new vertices can disappear and appear in the process, which may also be relevant for a football field. Provided the data issue (ensuring a sufficient volume of data for training an RL model with an attention mechanism) is resolved, the authors' methodology can be adapted to our case.

\subsection{Vehicle Routing Problem (VRP)}

Problem Statement of VRP:
The Vehicle Routing Problem (VRP) is an optimization problem aimed at finding the optimal routes for a fleet of vehicles, taking into account several factors such as time, cost, and load capacity. Formally, in VRP, there is a given number of vehicles, each with constraints on load capacity and working hours. There is also a set of client points that need to be visited, and for each point, the load and service time requirements are known. The objective of VRP is to optimally allocate the client points among the vehicles to minimize the total costs, considering all constraints.

Difference from TSP:
The main difference between TSP and VRP is that in TSP, there is one salesman who must visit a set of cities and return to the starting point, minimizing the total distance. In VRP, there are multiple vehicles, each needing to service a set of client points with specific constraints. Therefore, in VRP, it is necessary to optimize the allocation of client points among the vehicles considering various factors such as load capacity, service time, and throughput, making this problem more complex compared to TSP. The VRP problem statement may be relevant in the case of multiple observing cameras.

\subsection{CGN (Convolutional Graph Networks) for TSP}
The article \cite{gnn} discusses the use of GNN for optimizing the route when multiple cities need to be visited and returned to the starting point. The basics of the problem, the application of GNN to it, and the implementation details of the model using Graph Transformer and Residual Gated GCN are discussed. It concludes that the model demonstrates the ability to find optimal routes but requires labeled data, and a promising direction for development could be a transition to reinforcement learning. Labeled data are obtained using the Concorde TSP Solver, meaning that optimal paths for graphs need to be calculated first, and then the neural network training can begin.  
 
\chapter{Solution Description}
 {\color{red} This section is for decomposing the problem and describing the Methods for each subtask. }

 \section{Coordinate Systems Used}\label{41}
At this stage, two coordinate systems are used: the first is related to the football field, where the top-left corner is considered the point (0, 0), and the second is related to the camera coordinates, in a reference system linked to the far left corner of the field. For given coordinates $X,\;Y$, the transition from one plane to another is carried out according to the following law:

$$
\begin{bmatrix}
\tau_{i}X' \\
\tau_{i}Y' \\
\tau_{i}
\end{bmatrix} = 
\underbrace{ \begin{bmatrix}
a_{1} & a_{2} & b_{1} \\
a_{3} & a_{4} & b_{2} \\
c_{1} & c_{2} & 1
\end{bmatrix} }_{ M } \begin{bmatrix}
X \\
Y \\
1
\end{bmatrix},
$$

where $a_i$ are scaling/rotation elements, $\begin{bmatrix}
    b_2 & b_1
\end{bmatrix}^T$ is the shift vector, and $\begin{bmatrix}
    c_1 & c_2
\end{bmatrix}$ is the projection vector.

To find the true values of $X', Y'$, the resulting vector must be divided by the coefficient $\tau_i$, which is the scaling factor.

As a result, an algorithm was developed using the cv2 library, which takes two arguments as input: four initial coordinates (x, y), which indicate the initial corners of the rectangle of the transition area, and the second argument - four coordinates that indicate the desired corner coordinates for the output image. The algorithm produces a transformation matrix $M \in \mathbb{R}^{3 \times 3}$, after which the algorithm is sequentially applied to each frame from the initial dataset. As a result, a new dataset is obtained in which for each id, i.e., the recognized object in the frame, new coordinates ($X', Y'$) are obtained. An example visualizing the algorithm's work:

 \begin{figure}[!h]
     \centering
     \includegraphics[width=0.8\linewidth]{figures/Initial image.png}
     \caption{Image before transformation of coordinates}
     \label{fig:before-transform}
 \end{figure}

  \begin{figure}[!h]
     \centering
     \includegraphics[width=0.8\linewidth]{figures/Transformed Image with transformed player positions.png}
     \caption{Image after transformation of coordinates}
     \label{fig:after-transform}
 \end{figure}


\section{Preprocessing}
\section{Algorithm Development for the Case of Static Players}
There are many solutions for the TSP problem, but there are several key differences between them, the main one being the accuracy of the solution. Some algorithms can solve the problem of finding the minimal closed path exactly, while others do not guarantee the minimal path but are significantly faster. For this specific task, the exact solution option was chosen since the number $n$, i.e., the number of recognized objects in the frame, is small.

Thus, according to the article~\cite{Zhang_2021}, the most optimal option was the dynamic programming method, since among the exact approaches to solving TSP, it showed the highest speed. The implemented algorithm takes one frame with recognized objects, with already transformed coordinates, the transition to which was solved in section \ref{41}, and returns a path that includes the ids of all recognized objects in the frame, in the order that minimizes the Hamiltonian closed path in the graph, whose vertices are the ids, as well as the total path length. For the next task of developing an algorithm for the case of moving players, it will probably be necessary to revise the algorithmic approach to solving the problem. Nevertheless, the dynamic programming approach can be considered successful at this stage.

\section{Environment Simulation}
\subsection{Agent Simulation}
\subsection{Camera Simulation}
\subsubsection{Line - Plane Intersection}
Intersection of a plane with a line:
Let $p_0$ be the camera vector, and $v_0$ be the direction vector of the light ray passing through the camera vector (center of the focal lens).

Then the parametric equation of the line can be defined as:
$$P(t) = p_0 + t \cdot v_0 \qquad \text{($P(t)$ is a point on the line)}$$

We need to find such a $t$ that $P(t)$ lies on a given plane.
Substituting this into the plane equation, we get:
$$a(P0_x + tV0_x) + b(P0_y + tV0_y) + c(P0_z + tV0_z) = d$$

Solving for $t$ ($n$ is the normal vector of the plane):
$$t = \frac{D - nP0}{n \cdot V0}$$

We are interested in $t < 0$ since positive values correspond to the wrong light direction.

\subsubsection{Field of View and Angle of View}
For truthful simulation of field of view and physics of long-focus cameras, work of Matvey Gancev was used. As simulation of physics for camera is a complicated and time-requiring task, with the admition of Matvey Gancev, the simulation code was used. However modeling the camera movement and traversal algorithm are completed without use of intelectual property from other researchers. It is a complicated task to set the angle velocity, thus it can be estimated with a velocity on a euclidean plane. 

Field of View: The calculatePanoramicSystemFOV method calculates the angle of view (FOV) for each camera in the panoramic system and returns a list of camera angles of view for the panoramic system.

The method defines the plane vector and the coordinates and angles of the panoramic system. Then for each camera in the panoramic system, the following occurs:

\begin{enumerate}
    \item The camera coordinates and its angles of view are calculated.
    \item Vectors defining the camera's angles of view are created.
    \item Rotations are applied to the vectors of the camera and the panoramic system.
    \item The intersection point of the line (defined by the camera) with the plane (panoramic system) is calculated.
    \item The camera's angle of view and the main axis of the camera's view are calculated.
\end{enumerate}


1. Calculation of vector a:
$$
\text{vector\_a} = \begin{bmatrix}
1.0 \\
\tan\left(\frac{\text{camera\_width\_angle\_of\_view}}{2}\right) \\
-\tan\left(\frac{\text{camera\_height\_angle\_of\_view}}{2}\right)
\end{bmatrix}
$$

2. Rotation of vector $ \mathbf{a} $:
$$
\mathbf{a} = \left(\mathbf{R}_\text{ps} \cdot \mathbf{R}_\text{c}\right) \cdot \mathbf{a}
$$

Here $ \mathbf{R}_\text{ps} $ and $ \mathbf{R}_\text{c} $ are the rotation matrices for the panoramic system and the camera, respectively.

3. Re-rotation of vectors $ \mathbf{b} $, $ \mathbf{c} $, $ \mathbf{d} $, $ \mathbf{p} $:
\begin{align*}
\mathbf{b} &= \left(\mathbf{R}_\text{ps} \cdot \mathbf{R}_\text{c}\right) \cdot \mathbf{b} \\
\mathbf{c} &= \left(\mathbf{R}_\text{ps} \cdot \mathbf{R}_\text{c}\right) \cdot \mathbf{c} \\
\mathbf{d} &= \left(\mathbf{R}_\text{ps} \cdot \mathbf{R}_\text{c}\right) \cdot \mathbf{d} \\
\mathbf{p} &= \left(\mathbf{R}_\text{ps} \cdot \mathbf{R}_\text{c}\right) \cdot \mathbf{p} 
\end{align*}

Here $ \mathbf{R}_\text{ps} $ and $ \mathbf{R}_\text{c} $ are also the rotation matrices for the panoramic system and the camera.

4. Calculation of point $ \mathbf{t}_0 $:
$$
\mathbf{t}_0 = \mathbf{r}_\text{ps} + \mathbf{R}_\text{ps} \cdot \mathbf{r}_\text{c}
$$

Here $ \mathbf{r}_\text{ps} $ and $ \mathbf{r}_\text{c} $ are the coordinates of the panoramic system and the camera, respectively.

5. Definition of parameters $a_0$, $b_0$, $c_0$, $d_0$, $p_0$:

\begin{align*}
a_0 &= -\frac{\left(\mathbf{v}_\text{plane} \cdot \mathbf{v}_{\text{t}_0}\right) + D} {\left(\mathbf{v}_\text{plane} \cdot \mathbf{v}_a\right)} \\
b_0 &= -\frac{\left(\mathbf{v}_\text{plane} \cdot \mathbf{v}_{\text{t}_0}\right) + D} {\left(\mathbf{v}_\text{plane} \cdot \mathbf{v}_b\right)} \\
c_0 &= -\frac{\left(\mathbf{v}_\text{plane} \cdot \mathbf{v}_{\text{t}_0}\right) + D}{\left(\mathbf{v}_\text{plane} \cdot \mathbf{v}_c\right)} \\
d_0 &= -\frac{\left(\mathbf{v}_\text{plane} \cdot \mathbf{v}_{\text{t}_0}\right) + D}{\left(\mathbf{v}_\text{plane} \cdot \mathbf{v}_d\right)} \\
p_0 &= -\frac{\left(\mathbf{v}_\text{plane} \cdot \mathbf{v}_{\text{t}_0}\right) + D}{\left(\mathbf{v}_\text{plane} \cdot \mathbf{v}_p\right)}
\end{align*}

6. Calculation of the coordinates of the angles of view and the main axis of the camera:
\begin{align*}
\text{fov\_a} &= a_0 \cdot \mathbf{v}_a + \mathbf{v}_{\text{t}_0} \\
\text{fov\_b} &= b_0 \cdot \mathbf{v}_b + \mathbf{v}_{\text{t}_0} \\ 
\text{fov\_c} &= c_0 \cdot \mathbf{v}_c + \mathbf{v}_{\text{t}_0} \\ 
\text{fov\_d} &= d_0 \cdot \mathbf{v}_d + \mathbf{v}_{\text{t}_0} \\ 
\text{main\_axis} &= p_0 \cdot \mathbf{v}_p + \mathbf{v}_{\text{t}_0} \\
\end{align*}

These transformations are necessary to calculate the parameters of the angles of view and the main axis of the camera in the panoramic system. They perform transformations of coordinates and vectors, taking into account their initial positions and rotations relative to the coordinate system associated with the panoramic system. Then, using the found parameters, the points indicating the angles of view of each camera are determined, as well as the point representing the main axis of the camera. These steps allow us to determine the position and direction of view of each camera in the context of the panoramic system.


Angle of View: 
This method calculates the vertical angle of view of the camera. Let's consider the mathematics of this process.

Let $f$ be the focal length of the camera lens, and $H$ be the height of the camera's focal plane. We want to find the vertical angle of view $\theta_{\text{vertical}}$, which determines how many degrees vertically the camera covers.

Using the theorem of similar triangles, we understand that the vertical angle of view can be expressed as double the arctangent of the ratio of the height of the focal plane to twice the focal length:

$$
\theta_{\text{vertical}} = 2 \times \arctan \left( \frac{H}{2f} \right)
$$

Thus, we find the angle at which the image on the focal plane is visible relative to the central axis of the camera. This gives us an idea of what portion of the vertical space is covered by the image.

\subsubsection{Determining $\Delta$ Yaw and $\Delta$ Pitch during Movement}
To simulate the camera and the algorithm as a whole, it is important to have the ability to control the camera.
To control the camera in our study, we can send $\Delta \theta$ and $\Delta \phi$ (yaw and pitch changes) to our camera (simulated or real).

1. Central point of the field of view angles:

   Find the central point between two field of view (FOV) angles:

   Let $ P_1  $ and  $ P_2  $ be the two field of view angles of the camera, then the central point $ C  $ will be:

    $$
   C = \left( \frac{P_{1x} + P_{2x}}{2}, \frac{P_{1y} + P_{2y}}{2} \right)
    $$

2. Calculation of the angle:

   Calculate the tilt angle of the camera relative to the horizon and the vertical angle of view of the camera:

   Let $ (x_c, y_c)  $ be the coordinates of the camera, $ (x_i, y_i)  $ the initial position, and $ (x_t, y_t)  $ the target position. Also, $ h  $ is the height of the camera, and $ \theta_{\text{pitch}}  $ the vertical angle of view, $\theta_0$ the current vertical tilt angle, and $\theta_{\text{top\_aov}}$ the angle between the highest incoming light ray on the camera and the main axis of the camera ($AOV_{\text{vert}}/2$).

   First, determine the camera's yaw angle, using direction vectors from the camera to the initial and target positions:

    $$
   \Delta \theta_{\text{yaw}} = \text{atan2}(y_t - y_c, x_t - x_c) - \text{atan2}(y_i - y_c, x_i - x_c)
    $$

   Then calculate the pitch angle, which is the tilt angle of the camera relative to the horizon for:

    $$
   \theta_{\text{pitch}} = 90^\circ - \text{toDegrees}(\arctan( \frac{|(x_t-x_c, y_t-y_c) | }{h}) )
    $$

    $$
   \Delta \theta_{\text{pitch}} = \theta_{\text{pitch}} - \theta_{\text{top\_aov}} - \theta_{0}
    $$

Here, $ \text{atan2}  $ is the arctangent function of two arguments, $ \| \cdot \|  $ is the vector norm, and $\mod$ is the modulo operation.

\subsubsection{Player Detection inside of a Field of View}
It is already mentioned, that the algorithm is able to infer how close the center of FOV to the target position, but the algorithm also have a player detection module for future algorithm enhancements (for example: awaiting for the moment when player view is not obstructed by other players). 

\lstinputlisting[numbers=none]{listings/agent_in_fov.txt}


\section{Algorithm Development for the Case of Moving Players}
The future movement of players is assumed to be unknown.

\subsection{Master-Route Baseline}
The basic algorithm is to pass the camera over the field in a "snake" pattern. The camera sequentially surveys each strip of the observable field, moving from the left edge of the field to the right, and then from the right edge of the field to the left. This algorithm is easily implemented once the framework from section 4.4.2 has been implemented. To cover all players, the camera can be approximated along the path when the algorithm determines that the player is close enough to its initial territory.


\pagebreak
\subsection{KNN-Greedy Algorithm}

\lstinputlisting[numbers=none]{listings/knn_greedy.txt}

In the above fragment, the main algorithm is described. 

\subsection{Probability-Density-Graphs TSP}
Assume we know the player graph at time t. The features of the vertices in this situation will be the coordinates of the players as well as identifiers. We train an algorithm that can predict the probabilistic distribution of each player's position over the sequence $[0,t]$ (the simplest case being a 2-dimensional Gaussian, i.e., standard normal distribution), then the goal for the camera can be set to focus on the 95\% confidence interval at time t+k, which can be calculated using the camera's angular velocity. The camera then focuses on the vertex so that the entire area is visible with sufficient confidence and zooms in on a specific player (linear interpolation of their movement can be included here). The algorithm then repeats for all players.       




\chapter{Results}

Working on the research so far we have managed to
\begin{itemize}
    \item Get acquainted with the research topic through literature review
    \item Mark and pre-process the original data
    \item Develop the coordinate transformation for conversion of the coordinates of a camera to the view from above
    \item The results of the previous step were used for the solution of the static TSP problem 
    \item For all of the above steps the code has been written, and the model works, giving expected results
\end{itemize}

The next step would be the implementation of the dynamic TSP model and the publication of the results.

\section{Evaluation of Camera Traversal Algorithm }
\subsection{Metric and Experiment formulation}
To evaluate the algorithm, the metric chosen is time required to traverse through all players with stopping time on a player equal to $t_{stop} = 5$. Number of players $n_{players} = 22$ and $n_{iter} = 100$. An experiment will be ran $n_{iter}$ of times, and the T will be the metric for the according algorithm:
$$
T=\frac{1}{n_{iter}}\sum\limits_{i=1}^{n_{iter}} t_{i}
$$

Note that $t_{i}$ is measured in simulation ticks, with conversion formula $25  \cdot t_{i} = 1 \text{ second}$, giving FPS (frames per second) to be 25 ($FPS=25$). Also statistics like standard deviation and interquantile range will be displayed for further analysis. 

Given such framework of evaluation, it is possible to compare quality of a baseline to the quality of the developed KNN-Greedy approach. {\color{purple}(Possibly delete a part about comparing)}

\subsection{Dataset}
To assure that experiments are representative of the real-world scenario, a soccer match simulation from Timur Khaibrakhmanov was used. In the simulation, player movement is modelled with similar to real-world laws. Despite that simulation quite accurately represents football matches, it has a randomization component, that allows to create $n_{iter}=100$ diverse simulations, that represent the soccer dynamics. On those generated simulations, the algorithm is tested. Algorithm only posesses the information prior to during timestamp, thus it simulates real-world unpredictability of players' behaviour.

\subsection{Result of the Experiment}
\begin{table}[h!]
\centering
\caption{Statistics of the KNN Greedy Performance for $N=100$ in frames}
\begin{tabular}{lr}
\toprule
Statistic & Frames for Complete Traversal \\
\midrule
Count     & 100.00000 \\
\textbf{Mean}      & \textbf{408.56000} \\
Standard Deviation (std) & 46.73919 \\
Minimum (min)  & 305.00000 \\
25th Percentile (25\%) & 375.75000 \\
Median (50\%)  & 412.50000 \\
75th Percentile (75\%) & 442.25000 \\
Maximum (max)  & 549.00000 \\
\bottomrule
\end{tabular}
\label{table:algorithm_stats}
\end{table}



\begin{table}[h!]
\centering
\caption{Statistics of the KNN Greedy Performance for $N=100$ in seconds}
\begin{tabular}{lr}
\toprule
Statistic & Seconds for Complete Traversal \\
\midrule
Count     & 100.00000 \\
\textbf{Mean}      & \textbf{16.3424} \\
Standard Deviation (std) & 1.8695676\\
Minimum (min)  & 12.2 \\
25th Percentile (25\%) & 15.03 \\
Median (50\%)  & 16.5 \\
75th Percentile (75\%) & 17.69 \\
Maximum (max)  & 21.96 \\
\bottomrule
\end{tabular}
\label{table:algorithm_stats}
\end{table}




%\include{6-specification}
%\section{Спецификация оборудования}

Перечень оборудования, поставленного по   Муниципальному контракту №0350300011821000371\_175478 от 04.10.2021г. на поставку серверного оборудования для подсистемы видеонаблюдения АПК «Безопасный город», серверного оборудования к существующей инфраструктуре Заказчика приведен в таблице \ref{tab:spec}.

\begin{landscape} 
\begin{center}
\begin{longtable}{|l|l|l|l|}
\caption{Спецификация оборудования поставленного по   Муниципальному контракту №0350300011821000371\_175478 от 04.10.2021г.} \label{tab:spec}  \\

\hline \multicolumn{1}{|c|}{\textbf{№}} & \multicolumn{1}{p{110pt}|}{\textbf{Наименование }} & \multicolumn{1}{p{160pt}|}{\textbf{Спецификация}}  & 
\multicolumn{1}{p{70pt}|}{\textbf{Имя и IP}} \\ \hline 
\endfirsthead

\multicolumn{4}{p{340pt}}%
{{\bfseries \tablename\ \thetable{} -- продолжение}} \\
\hline \multicolumn{1}{|c|}{\textbf{№}} & \multicolumn{1}{p{110pt}|}{\textbf{Наименование }} & \multicolumn{1}{p{160pt}|}{\textbf{Спецификация}} & 
\multicolumn{1}{p{70pt}|}{\textbf{Имя и IP}}  \\ \hline 
\endhead

\hline \multicolumn{4}{|r|}{{Продолжение на сл.странице}} \\ \hline
\endfoot

\hline \hline
\endlastfoot

1 & МВЯ -1 & Intel  i9 10900x, ddr4 32gb, sad 120gb,sfp+ 10gbit, RTX3060ti x2  & MPC01  194.255.255.0 \\\hline
2 & МВЯ -2 & Intel  i9 10900x, ddr4 32gb, sad 120gb,sfp+ 10gbit, RTX3060ti x2    & MPC02 \\\hline
3 & МХВ -1 &Intel   corei7 -9700,ОЗУ 16 Gb , 1хssd 120g+12х6тб, sfp+ 10gbit   & SS01\\\hline
4 & МХВ -2 & Intel   corei7 -9700,ОЗУ 16 Gb , 1хssd 120g+12х6тб, sfp+ 10gbit  & SS02\\\hline
5 & МХВ -3 & Intel   corei7 -9700,ОЗУ 16 Gb , 1хssd 120g+12х6тб, sfp+ 10gbit   & SS03\\\hline
6 & МХВ -4 & Intel   corei7 -9700,ОЗУ 16 Gb , 1хssd 120g+12х6тб, sfp+ 10gbit   & SS04\\\hline
7 & МХВ -5 &Intel   corei7 -9700,ОЗУ 16 Gb , 1хssd 120g+12х6тб, sfp+ 10gbit    & SS05\\\hline
8 & МХВ -6 & Intel   corei7 -9700,ОЗУ 16 Gb , 1хssd 120g+12х6тб, sfp+ 10gbit   & SS06\\\hline
9 & Коммутатор ТШ-2 & MikroTik CRS317  & MTIK317 \\\hline
10 & KVM & Aten & \\\hline
11  & Коммутатор ТШ-1  &MikroTik CRS326   & MTIK326\\\hline
12 & ИБП 10000   & IPPON 10000VA
SNMP card  & \\\hline
13  & ТШ-2   &   & \\\hline
 
\end{longtable}
\end{center}
\end{landscape}  

%\include{7-port-comm}

\backmatter %% Здесь заканчивается нумерованная часть документа и начинаются ссылки и
            %% заключение

%\include{7-conclusion}  % Выводы 

\include{81-biblio}      % Список литературы 

%\include{98-Otzyv}

%\appendix   % Тут идут приложения

%\include{91-appendix2} 

% %\chapter*{Перечень внесенных изменений}
\Large{Лист изменений}
\small
\begin{longtable}{|p{90pt}|p{100pt}|p{260pt}|p{80pt}|}
 
\hline
\textbf{Дата} & \textbf{Автор} & \textbf{Внесенные изменения.}   \\
\hline
\endfirsthead
\multicolumn{4}{c}%
{\tablename\ \thetable\ -- \textit{Продолжение}} \\
\hline
\textbf{Дата} & \textbf{Автор} &\textbf{Внесенные изменения.}   \\
\hline
\endhead
\hline \multicolumn{4}{r}{\textit{Продолжение на следующей странице}} \\
\endfoot
\hline
\endlastfoot
  	 &   &    \\\hline
 	 &   &   \\  \hline
 	 & &    \\\hline
 	 & &    \\\hline
 	 & &    \\\hline 
 	 & &    \\\hline
 	 & &    \\\hline
 	 & &    \\\hline
 	 & &    \\\hline
 	 & &    \\\hline
 	 & &    \\\hline
 	 
\end{longtable}



\end{document}

%%% Local Variables:
%%% mode: latex
%%% TeX-master: t
%%% End:

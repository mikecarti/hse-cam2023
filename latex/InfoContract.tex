

1.1.Сведения о работе
Во исполнение постановления Правительства Российской Федерации от 25 апреля 2012 г. № 394 "О мерах по совершенствованию использования информационно-коммуникационных технологий в деятельности государственных органов", постановления Правительства Российской Федерации от 25 октября 2013 г. № 959  "О Федеральном агентстве научных организаций" в рамках реализации Плана информатизации государственного органа  на 2016 финансовый год и плановый период 2017 и 2018 годы, утвержденного приказом Федерального агентства научных организаций от 8 августа 2016 г. № 404, в Федеральном агентстве научных организаций должны быть проведены работы (оказаны услуги) по созданию информационной системы «Аналитическая система ФАНО России».
1.2. Принятые обозначения и сокращения
В настоящем техническом задании используются следующие обозначения и сокращения, приведенные далее в таблице 1.

Наименование сокращения (обозначения)
Расшифровка сокращения (обозначения)
2
3
CSV
CSV (от англ. Comma-Separated Values — значения, разделённые запятыми) — текстовый формат, предназначенный для представления табличных данных
DHTML
Dynamic HTML – это набор средств, которые позволяют создавать более интерактивные Web-страницы без увеличения загрузки сервера
HTML
Язык гипертекстовой разметки документов (от англ. Hypertext Markup Language – “язык гипертекстовой разметки”)
HTTP
Протокол прикладного уровня для передачи данных, используемый в Web (от англ. HyperText Transfer Protocol - «протокол передачи гипертекста») 
IP-адрес
Уникальный сетевой адрес узла в компьютерной сети, построенной по протоколу IP
JavaScript
Прототипно-ориентированный сценарный язык программирования. Наиболее широкое применение находит в браузерах как язык сценариев для придания интерактивности веб-страницам
JPEG (JPG)
JPEG - один из популярных графических форматов, применяемый для хранения фотоизображений и подобных им изображений. Файлы, содержащие данные JPEG, обычно имеют расширения .jpg, .jfif, .jpe или .jpeg.
MS SQL
Microsoft SQL Server — система управления реляционными базами данных (РСУБД), разработанная корпорацией Microsoft
PDF
Portable Document Format (PDF) — межплатформенный формат электронных документов, разработанный фирмой Adobe Systems
PHP
Cкриптовый язык общего назначения, интенсивно применяемый для разработки веб-приложений.
PNG
Растровый формат хранения графической информации, использующий сжатие без потерь качества
Windows Server 
Линейка серверных операционных систем от компании Microsoft
Агентство, ФАНО России
Федеральное агентство научных организаций 
Государственный контракт
Государственный контракт на выполнение работ, в рамках которого сформировано настоящее техническое задание
Заказчик
Федеральное агентство научных организаций 
НСИ
Нормативно – справочная информация
АС 
Аналитическая система ФАНО России
Интернет
Информационно-телекоммуникационная сеть Интернет
Исполнитель
Лицо, с которым ФАНО России заключило Государственный контракт
ИТ
Информационные технологии, информационно-технологический
Открытые данные
Информация, размещаемая ее обладателями в сети «Интернет» в формате, допускающем автоматизированную обработку без предварительных изменений человеком в целях повторного ее использования
Официальный сайт
Информационная система «Информационный портал ФАНО России (официальный сайт)»
ПО 
Программное обеспечение
СМЭВ
Система межведомственного электронного взаимодействия
ФОИВ
Федеральный орган исполнительной власти
Опытный образец
Опытный образец, эскизный экземпляр создаваемой Системы




1.3.Цель и задачи

1.3.1 Цель создания системы
Целью работы является создание единой централизованной ИС, автоматизирующей деятельность сотрудников в центральном аппарате и территориальных органах ФАНО России, в обязанности которых входит получение, анализ и сопоставление данных о деятельности подведомственных учреждений. Создаваемая информационная система обеспечит получение, анализ и сопоставление данных, хранение результатов сопоставления данных. 
В динамике современных условий увеличения требований к скорости анализа большого объема информации для обеспечения высокого качества принятии управленческих решений руководством ФАНО России наличие систем, обеспечивающих интерактивное отображение информации, является первостепенной задачей.

1.3.2 Задачи проекта
Задачей, выполняемого в 2016 году проекта в соответствии с настоящим техническим заданием, является создание опытного образца АС ФАНО России.
Опытный образец АС ФАНО России должен обеспечить решение следующих задач: 
Создание механизма получения актуальной информации из информационных систем ФАНО России (в части получения консолидированной информации, о различных аспектах деятельности организаций подведомственных ФАНО России Объем и уровень агрегации данной информации является задачей предпроектного обследования;
Создание структуры данных и механизма отображения  актуальной информации о деятельности организаций, подведомственных ФАНО России из открытых источников и других государственных и ведомственных систем Российской Федерации (www.bus.gov.ru, Web Of Science, Scopus, РИНКЦЭ, www.sciencemon.ru, Роспатент, Реестр юридических лиц, ЕГРЮЛ, www.fedresurs.ru);
Обеспечение сбора и первичной обработки информации, необходимой для подготовки отчетности по показателям деятельности организаций, подведомственным ФАНО России;
Создание отдельного хранилища информации для ее дальнейшего анализа;
Создание механизма формирования и представления аналитической информации о деятельности организаций, подведомственных ФАНО России;
Создание настраиваемого web интерфейса для отображения информации о деятельности организаций, подведомственных ФАНО России в том числе с использованием ГИС в интересах поддержки принятия решений руководством Агентства;
Создание механизма формирования аналитической отчетности в электронном и печатном виде;
Создание механизма обеспечения экспертной деятельности экспертных советов и внешних экспертов при оценке деятельности организаций, подведомственных ФАНО России. 
Создание настраиваемого web интерфейса для отображения информации в интересах экспертных советов и внешних экспертов при оценке результативности деятельности научных организаций.

1.4. Характеристика объекта автоматизации
Федеральное агентство научных организаций является федеральным органом исполнительной власти, осуществляющим функции по нормативно-правовому регулированию и оказанию государственных услуг в сфере организации деятельности, осуществляемой подведомственными организациями, в том числе в области науки, образования, здравоохранения и агропромышленного комплекса, а также по управлению федеральным имуществом организаций, подведомственных ФАНО России. ФАНО России осуществляет функции и полномочия учредителя и собственника федерального имущества, закреплённого за подведомственными ему организациями. Руководство деятельностью Федерального агентства научных организаций осуществляет Правительство Российской Федерации.


Структурное подразделение
Наименование процесса
Управление методологии
Анализ отклонений фактических значений показателей деятельности организаций, подведомственных ФАНО России от плановых 

Анализ исполнения государственных заданий учреждений, подведомственных ФАНО России

Анализ финансово-хозяйственной деятельности организаций, подведомственных ФАНО России; 
Анализ эффективности использования имущественного комплекса;
Анализ размещения государственного и исполнения государственного заказа.
Управление академического взаимодействия и обеспечения деятельности Научно-координационного совета

Оценка результативности деятельности научных организаций, подведомственных ФАНО России

Экспертная деятельность экспертных советов и внешних экспертов


1.5.Использование и применение нормативных правовых актов 
Основанием для выполнения работ (оказанию услуг) по Государственному контракту являются следующие нормативные правовые акты:
Федеральный закон от 9 февраля 2009 г. № 8-ФЗ «Об обеспечении доступа к информации о деятельности государственных органов и органов местного самоуправления»;
Указ Президента Российской Федерации от 7 мая 2012 г. № 601 «Об основных направлениях совершенствования системы государственного управления»;
постановление Правительства Российской Федерации от 24 ноября 2009 г. № 953 «Об обеспечении доступа к информации о деятельности Правительства Российской Федерации и федеральных органов исполнительной власти»;
распоряжение Правительства Российской Федерации от 30 января 2014 г. № 93-р об утверждении Концепции открытости федеральных органов исполнительной власти;
постановление Госстандарта СССР от 24 марта 1989 г. № 661 «ГОСТ 34.602-2001. Информационная технология. Комплекс стандартов на автоматизированные системы. Техническое задание на создание автоматизированной системы.»;
постановление Госстандарта СССР от 4 сентября 2001 г. № 367-ст «ГОСТ 7.32-2001. Межгосударственный стандарт. Система стандартов по информатизации, библиотечному и издательскому делу. Отчет о научно-исследовательской работе. Структура и правила оформления»;
приказ Минэкономразвития России от 16 ноября 2009 г. № 470 «О Требованиях к технологическим, программным и лингвистическим средствам обеспечения пользования официальными сайтами федеральных органов исполнительной власти» (Зарегистрирован Минюстом России 31 декабря 2009 г., регистрационный № 15949);
приказ Минкомсвязи России от 30 ноября 2015 г. № 483 «Об установлении Порядка обеспечения условий доступности для инвалидов по зрению официальных сайтов федеральных органов государственной власти, органов государственной власти субъектов Российской Федерации и органов местного самоуправления в сети «Интернет» (Зарегистрирован Минюстом России 1 февраля 2016 г., регистрационный № 40905).


1.
Этап № 1. Подготовка технического задания на информационную систему
1.1
Проведение предпроектного обследования:
1. Проведение  предпроектного обследования и подготовка отчета  
Отчет должен включать:  
- детализацию технических требований к платформе для разработки решения;
- детализацию требований к работам, выполняемым в рамках создания опытного образца АС;
- детализацию функциональных требований к АС;
- детализацию требований к производительности и надежности АС;
- детализацию  требований к составу и параметрам технических средств АС;
- детализацию  требований к составу программных средств (хранилищу документов;  серверу приложений; платформе управления информацией и отображением; серверу публикации; серверу аналитики; библиотекам клиентского интерфейса; платформе интеграционного модуля).

2. Обобщение требований, сформированных в рамках предпроектного обследования, разработка документа «Техническое задание на разработку АС» и согласование его с Заказчиком.
1.2
Разработка программы и методики испытаний  АС 
Этап № 2. Предоставление неисключительных прав на программное обеспечение
2.1
Предоставление  неисключительных прав на программное обеспечение технологической платформы информационно – аналитической системы (требования к функциональным возможностям приведены в разделе 4)
Этап № 3. Установка, настройка и монтаж средств программного обеспечения


3.1
Установка, настройка и монтаж средств программного обеспечения:
1. Установка, настройка и монтаж средств программного обеспечения
2. Разработка комплекта документации опытного проекта АС 
3. Разработка опытного образца АС ФАНО России
Этап № 4. Проведение предварительных испытаний
4.1
Проведение предварительных испытаний опытного образца АС для принятия решения о разработке информационной системы на выбранной платформе


Раздел 3.Требования к характеристикам опытного образца Аналитической системы 
3.1. Общие требования по структуре системы
Аналитическая система должна включать следующие компоненты:
подсистему управления объектно-ролевым доступом;
подсистему сбора, обработки и загрузки данных (интеграционный модуль);
подсистему хранения данных;
подсистему формирования и хранения документов в электронном виде;
подсистему управления реестрами и классификаторами;
подсистему обеспечения экспертной работы; 
подсистему анализа данных;
подсистему отображения информации.
 
3.2. Функциональные требования к опытному образцу АС 
Опытный образец АС должен обеспечивать работу следующих пользователей: сотрудников ФАНО России (специалистов, руководителей, администраторов), членов комиссий, создаваемых для выработки решений по программам, проектам, конкурсам (включая Ведомственную комиссию ФАНО России по оценке результативности деятельности подведомственных организаций), членов экспертных советов (далее - ЭС), внешних экспертов. 
Опытный образец АС должен обеспечивать:
загрузку и анализ информации из информационных систем ФАНО России;
хранение и анализ информации о деятельности научных организаций из внешних источников (Web Of Science, Scopus, РИНКЦЭ, Роспатент);
обеспечение информационного обмена с другими ФОИВ через СМЭВ с целью получения сведений о деятельности научных организаций;
создание и управление настраиваемым web интерфейсом для отображения информации в интересах поддержки принятия решений руководством Агентства;
отображение информации в интересах экспертных советов и внешних экспертов при оценке результативности деятельности научных организаций;
формирование наглядной аналитической отчетности в электронном и печатном виде;
обеспечение обмена информацией в рамках деятельности экспертных советов и внешних экспертов.
3.3. Требования по назначению
АС должна обеспечивать возможность исторического хранения данных с неограниченной глубиной.
АС должна обеспечивать возможность одновременной работы до 500 пользователей при следующих характеристиках времени отклика системы для документов, не превышающих  1000 Кб:
для операций навигации по экранным формам системы  – не более 5 сек;
для операций сохранения данных электронных документов и формирования отчетов  – не более 15 сек, если для конкретной операции или документа (отчета) явно не предусмотрено обратное.
АС должна предусматривать возможность масштабирования по производительности и объему обрабатываемой информации без модификации ее программного обеспечения путем модернизации используемого комплекса технических средств. Возможности масштабирования должны обеспечиваться средствами используемого базового программного обеспечения. 

3.4.  Требования к функциональным модулям АС 

3.4.1. Требования к  подсистеме управления объектно-ролевым доступом
Подсистема должна обеспечивать выполнение следующих функций:
управление учетными записями пользователей АС;
управление группами пользователей;
управление ролями и правами пользователей:
формирование отчета об истории изменений назначенных полномочий за период;
формирование отчета о полномочиях назначенных пользователю АС;
формирование отчета о полномочиях назначенных группе пользователей  АС.
Подсистема должна хранить учетные записи пользователей, информацию о группах пользователей  в службе каталогов.

3.4.2. Требования к интеграционному модулю
Интеграционный модуль должен обеспечивать выполнение следующих функций:
управление информационным обменом между подсистемами АС;
управление информационным обменом между АС и внешними информационными системами, включая описания настроек источников и приемников данных, направление потоков обмена, их состав и расписание выполнения, а также ведение журналов выполненных операций;
информационный обмен посредством электронной почты, файловых ресурсов, веб-сервисов;
обеспечение сбора данных посредством ввода информации в формы ввода;
обеспечение сбора данных из открытых российских и международных информационных источников;
обеспечение первичного контроля входной информации, включающего проверку состава информации, ее полноту, типизацию, соответствие НСИ;
обеспечение контроля качества информации, размещаемой в хранилище, при котором должно быть обеспечено своевременное обнаружение нарушений качества информации;
обеспечение информационного обмена  с органами исполнительной власти регионов и Российской Федерации в части получения сведений с использованием  инфраструктуры электронного правительства (СМЭВ, Единого Портала Госуслуг и других).
Интеграция АС с информационными системами ФАНО России в рамках Государственного контракта предусматривает, в том числе, автоматический интерфейс выгрузки информации из АС в информационную систему «Информационный портал ФАНО России (официальный сайт)». Подробные требования к интеграции приведены в подпункте 3.4.2. настоящего технического задания.
Разрабатываемые функциональные возможности не должны препятствовать интеграциям с государственными информационными системами в рамках иных контрактов.

3.4.3. Требования к подсистеме хранения данных
Подсистема хранения данных должна обеспечивать хранение всей информации о подведомственных организациях в виде, удобном для построения аналитической отчетности. 
Хранение должно осуществляться в реляционной СУБД за исключением событий жизненного цикла подведомственных организаций, для которых должно быть предусмотрено хранение в нереляционной базе данных Cassandra 2.0 или аналогичной.

3.4.4. Требования к подсистеме формирования и хранения  документов в электронном виде
Подсистема анализа и хранения документов должна обеспечивать следующий функционал:
Хранение документов по подведомственным организациям в виде файлов и метаданных к ним;
Доступ к файлам с использованием CMIS интерфейса;
Формирование образов XML документов в PDF формате на основе XSL шаблонов, хранящихся в системе;
Интеграция в состав формируемых документов графиков, диаграмм, таблиц в форматах SVG, PNG, EPS, TIFF;
Персонализованное управление доступом к каталогам и отдельным файлам в соответствии с назначенными политиками;
Доступ к файлам по HTTPS ссылке. 

3.4.5. Требования к подсистеме управления реестрами и классификаторами
Основу процесса ведения реестров и классификаторов АС представляет множество справочников ведомственного, отраслевого, межотраслевого, общегосударственного и международного уровня.
Все справочники, находящиеся в сети интернет в свободном доступе, должны быть скопированы в АС.
Для таких справочников должна быть реализована процедура их поддержания в актуальном состоянии на основе мониторинга изменений в структуре и наполнении справочников.
Допустимо не создавать копий справочников и классификаторов, используемых в АС и входящих в состав информации, перечисленной в реестре НСИ ФГИС ЕСНСИ (http://nsi.gosuslugi.ru/_layouts/NsiInfrastructure/WelcomePage.aspx).
В административном интерфейсе АС должны быть реализованы следующие основные регистры хранения справочной информации: 
справочник пользователей АС; 
справочник экспертов АС.
В административном интерфейсе АС  должны быть реализованы следующие основные реестры хранения справочной информации: 
справочник категорий административных функций;
справочник подведомственных организаций.
При этом в составе подсистемы управления реестрами и классификаторами должны быть реализованы следующие функции:
ведение информации о реорганизации подведомственных организаций;
ведение информации о структурных подразделениях организаций, подведомственных ФАНО России;
работа с семантическим графом, описывающим предметные области, в которых работают подведомственные ФАНО России организации и их структурные подразделения. 

3.4.6. Требования к подсистеме обеспечения экспертной работы
Подсистема обеспечения экспертной работы должна обеспечивать следующий функционал: 
ведение реестра экспертов, включая информацию об их специализациях;
ведение реестра экспертных советов;
подготовку вопросника для экспертизы;
рассылку вопросника экспертам по заданному списку или на основе сформулированных критериев;
получение результатов экспертизы от экспертов в электронном виде;
формирование сводного отчета по результатам экспертизы. 

3.4.7. Требования к подсистеме анализа данных
Подсистема должна обеспечивать выполнение следующих функций:
обновление метрик и показателей при каждом обновлении исходной информации (общее количество показателей и метрик:  не менее 100 и не более 10 000);
формирование наглядного представления информации в виде графиков, диаграмм, таблиц;
формирование аналитических слоев для ГИС;
формирование аналитических отчетов по запросу или по регламенту;
формирование отчетов по информации, получаемой от подведомственных организаций и третьих сторон;
подготовку информации для ее включения в печатные и интерактивные аналитические компоненты в форматах XML и JSON.

3.4.8. Требования к подсистеме отображения информации
Подсистема отображения информации совместно с подсистемой управления объектно-ролевым доступом предназначена для централизованного управления отображением информации на индивидуальных и коллективных средствах отображения.
Основные требования к системе отображения должны реализовываться средствами используемого программного обеспечения технической платформы и  приведены в разделе 4.

Раздел 4. Требования к программному обеспечению технической платформы 
Программное обеспечение технической  платформы, неисключительные права на которое передаются Заказчику, должно обеспечивать следующий набор функций:
мониторинг объектов по состоянию;
управление потоком событий по объектам управления;
управление набором метрик и показателей;
получение и анализ информации из информационных баз;
формирование и анализ формальных фактов по  объектам управления;
управление отображением информации. 

4.1 Требования к мониторингу объектов по состоянию
В части мониторинга объектов по состоянию программное обеспечение технической  платформы  должно обеспечивать следующий набор функций:
ведение перечня объектов мониторинга (общее количество объектов мониторинга  не менее 1000 и не более 5000);
ведение перечня состояний жизненного цикла объектов управления (организаций подведомственных ФАНО России);
анализ изменений в жизненном цикле объектов управления при получении информации о событиях;
настройка оповещений и уведомлений должностных лиц при изменении состояний жизненного цикла.

4.2 Требования к управлению потоком событий по объектам управления
В части управления потоком событий по объектам управления программное обеспечение технической  платформы  должно обеспечивать следующий набор функций:
ведение перечня типов событий;
задание дополнительных атрибутов событий различных типов;
отображение событий в виде отметок на картооснове и в виде списка;
детализированное отображение информации о событии;
автоматическое объединение событий в группы;
привязка типового реагирования к типам событий.

4.3 Требования к управлению набором метрик и показателей
В части управления набором метрик и показателей программное обеспечение технической  платформы  должно обеспечивать следующий набор функций:
ведение реестра используемых для описания подведомственных организаций типов показателей и метрик (с общим количеством показателей и метрик в соответствии с  подпунктом 3.4.7. Требований к подсистеме анализа данных);
перерасчет значений показателей и метрик при получении информации о событии жизненного цикла организаций, влияющих на значения показателей и метрик.

4.4 Требования к получению и анализу информации из баз Web Of Science и Scopus, РИНЦ
В части мониторинга объектов по состоянию программное обеспечение технической  платформы  должно обеспечивать следующий набор функций:
техническую возможность подключения к информационным базам  Web Of Science и Scopus, РИНЦ;
хранение и отображение информации о публикационной активности сотрудников организаций, подведомственных ФАНО России;
выявление синонимов для названий организаций через механизмы анализа данных;
расчет сводных показателей и метрик, описывающих публикационную активность.

4.5 Требования к формированию и анализу формальных фактов по  объектам управления  
В части мониторинга объектов по состоянию программное обеспечение технической  платформы  должно обеспечивать следующий набор функций:
формирование перечня формальных логических  высказываний принимающих значения   «ИСТИНА» или «ЛОЖЬ»;
проверку множества логических высказываний на непротиворечивость (количество анализируемых логических высказываний:  от 3 до более 10000);
выдачу контрпримера при наличии противоречий в множестве логических высказываний;
формирование события, описывающего обнаружение некоторого события.

4.6 Требования к управлению отображением информации. 
В части управления отображением информации программное обеспечение технической платформы должно обеспечивать выполнение следующих функций:
настройка типовых шаблонов отображения информации;
настройка типовых шаблонов и сценариев отображения;
управление экранами;
управление окнами. 
В части управления экранами программное обеспечение технической платформы должно обеспечивать выполнение следующих функций:
создание нового типа экрана;
создание нового экрана (коллективного, индивидуального, мобильного);
привязка экрана к конкретному физическому устройству;
передача управления экраном другому пользователю.
Пользователь должен иметь возможность запомнить на сервере текущее состояние экрана и впоследствии открыть этот экран. 
В части управления окнами программное обеспечение технической платформы должно обеспечивать выполнение следующих функций:
раскрыть на весь экран;
свернуть в преднастроенный размер и положение на экране;
передача копии окна из экрана на другой экран;
передача клона окна из экрана на другой экран;
передача управления окном другому пользователю;
переместить на передний план;
переместить на задний план;
присвоить значение уровня;
создать копию или клон окна.

4.7 Требования к платформе функционирования программного обеспечения технической платформы
Программное обеспечение технической платформы должно функционировать в программно – аппаратной среде АС, описанной в Разделе 6.

Раздел 5. Общесистемные требования к опытному образцу АС ФАНО
5.1.Требования к функции администрирования  
Для администрирования АС должен использоваться административный интерфейс управления АС (далее – административный интерфейс).  
Разграничение прав доступа должно осуществляться путем управления списком пользователей, допущенных к работе в системе.
Административный интерфейс должен обеспечивать возможность логирования действий пользователей.
Система управления АС должна удовлетворяет следующим требованиям безопасности при доступе к административному интерфейсу:
аутентификация администраторов для работы с административным веб-интерфейсом по индивидуальным логинам (системным именам) и паролям;
административный веб-интерфейс АС включает в себя раздел, обеспечивающий управление списком администраторов; 
добавление и удаление учетных записей администраторов, назначение и изменение паролей;
ввод, просмотр и редактирование фиксированного набора контактных данных для каждого администратора;
блокировка учетной записи администратора без удаления из базы данных;
управление правами доступа.

5.2.Назначение полномочий
Управление доступом пользователей в системе должно осуществляться путем привязки к учетной записи пользователя к перечню назначенных полномочий. 

5.3.Требования к функции ведения справочников и классификаторов
Административный интерфейс АС должен предоставлять возможность управления справочниками и классификаторами, используемыми в работе системы.

5.4. Требования к контролю данных
При работе с системой должна быть предусмотрена возможность контроля заполнения данных полей форм в соответствии с требованиями к информации, вносимой в данные поля.

5.5.Требования к защите информации от несанкционированного доступа.
Исполнитель должен обеспечить следующие мероприятия по защите информации АС:
5.5.1 Защиту административного интерфейса АС от несанкционированного доступа:
контроль уникальности для каждого пользователя логина и пароля доступа к административному интерфейсу;
поддержку работы модуля журналирования успешных и неуспешных аутентификаций в административном интерфейсе;
контроль процесса аутентификации, с автоматической блокировкой учетной записи после заданного количества неудачным попыток аутентификации;
регламентную смену паролей пользователей, допущенных к администрированию программного обеспечения АС;
автоматическую регистрацию действий пользователей, допущенных к администрированию программного обеспечения АС;
5.6.1  Разграничение прав доступа к административным функциям путем управления списком пользователей, допущенных к администрированию программного обеспечения АС:
создание новой учетной записи;
удаление учетной записи;
редактирование личных данных учетной записи администратора;
блокировка учетной записи без ее удаления;
определение прав доступа для каждой учетной.

5.7.Доработка АС  ФАНО России без участия Исполнителя
После окончания действия Государственного контракта Заказчик должен иметь правовую и программно-техническую возможность самостоятельной доработки АС  без участия Исполнителя (далее — доработка).
Доработка будет, осуществляется только для нужд Заказчика.
Доработка будет, осуществляется Заказчиком самостоятельно либо путем заключения договора (контракта) на выполнение работ (оказание услуг).
Доработанный вариант системы предназначен для нужд Заказчика и не может быть передан кому-либо для коммерциализации, равно как не могут быть переданы права на него, в том числе лицам, осуществившим доработку.

Раздел 6. Технические требования к программно-аппаратному обеспечению  
Программное обеспечение АС  должно обеспечивать корректную работу при размещении на программно-аппаратной платформе со следующими характеристиками*: 
операционная система — AltLinux версии не ниже 6.0; 
сервер баз данных — MS SQL версии не ниже 5.5 / PostgtreSQL версии не ниже 9.4, Cassandra версии не ниже 2.0;
Перечень иных программных средств, используемых при развертывании АС, должен быть уточнен Исполнителем при разработке Технического задания.
В качестве аппаратной платформы должны использоваться виртуальные серверы из числа развернутых на площадке Заказчика общим числом от 1 до 8 с процессорами c тактовой частотой 2,2 GHz, суммарным количеством ядер процессоров всех серверов до 16 и суммарной оперативной памятью до 64 Gb.
* указание на товарный знак относится к оборудованию и программному обеспечению Заказчика, и приведено с целью более точного и четкого описания характеристик объекта закупки, а также в целях обеспечения взаимодействия поставляемых товаров с оборудованием Заказчика и исключения случаев их несовместимости.

Раздел 7. Требования к проведению испытаний опытного образца АС  ФАНО России
В рамках работ 2016 года предусмотрены следующие обязательные виды испытаний: предварительные испытания.
Предварительные испытания проводятся в соответствии с документом «Программа и методика испытаний».
Допускается проведение предварительных испытаний на специальных испытательных стендах Исполнителя. В ходе предварительных испытаний определяется готовность компонентов АС к запуску в опытную эксплуатацию.
Для проведения Испытаний должна быть организована комиссия в составе представителей Исполнителя, представителей Заказчика.
В ходе предварительных испытаний  компонентов АС  члены испытательной комиссии должны удостовериться в соответствии предъявленных результатов работ требованиям технического задания и идентичности предъявленных решений их описанию в документации.
Для этого должны быть выполнены все проверки функциональности, дана оценка предъявленным документам, дано заключение о возможности и готовности проведения опытной эксплуатации.
В случае выявления при проведении предварительных испытаний неисправностей Исполнитель должен устранить выявленные неисправности и внести изменения в сопроводительную документацию.
Число проводимых повторных предварительных испытаний не ограничено. При этом за нарушение сроков завершения этапов Государственного контракта Исполнитель несет ответственность согласно условиям Государственного контракта.

Раздел 8. Гарантийное обслуживание
Гарантийный срок обслуживания должен составлять не менее 18 (восемнадцати) месяцев с момента подписания сторонами окончательного (сводного) акта выполненных работ или акта об устранении замечаний.
В период гарантийного срока Исполнитель обязуется за свой счёт обеспечить:
1) устранение программных ошибок (сбоев), связанных с невыполнением или некорректным выполнением функций опытного образца АС  ФАНО России, предусмотренных настоящими техническими требованиями в срок до 5 календарных дней с момента заявления о выявленных ошибках, за исключением программных ошибок (сбоев), вызванных некорректными действиями пользователей, администраторов или третьих лиц;
2) предоставление контактной информации для связи (по электронной почте, телефону, факсу).
Объем предоставления гарантий качества работ: на весь объем выполненных работ.



№ п/п
Наименование товаров указание на товарный знак (его словесное обозначение) (при наличии), фирменное наименование (при наличии), наименование страны происхождения товара
Единица 
измерения
Кол-во
Цена за ед. изм., вкл. все налоги и другие обязательные платежи в соответствии с законодательством России (руб.)
1.
Работы (услуги) по подготовке технического задания на информационную систему
шт
1
2099790     
2.
Неисключительные права на программное обеспечение технологической платформы информационно – аналитической системы, соответствующее требованиям к функциональным возможностям приведенным в разделе 4 Технического задания (Часть VI  Конкурсной документации). Наименование программного обеспечения: COS.OC;
наименование страны происхождения: Российская Федерация
шт
1
3149685     
3.
Работы (услуги) по установке, настройке и монтажу средств программного обеспечения
шт
1
1049895     
4.
Работы (услуги) по проведению предварительных испытаний
шт
1
699930     




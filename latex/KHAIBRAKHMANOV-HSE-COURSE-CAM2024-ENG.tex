%% Преамбула TeX-файла

% 1. Стиль и язык
\documentclass[utf8x]{G7-32} % Стиль (по умолчанию будет 14pt)
%\documentclass{book} % Стиль (по умолчанию будет 14pt)
\usepackage[T2A]{fontenc}
\usepackage[russian]{babel}

% Остальные стандартные настройки убраны в preamble.inc.tex.

\usepackage{pdflscape}
\usepackage{booktabs}
\usepackage{svg}
\usepackage{longtable}
\usepackage{pdflscape}
\usepackage{lscape}
\usepackage{cancel}
\usepackage{wrapfig}
% Настройки листингов.
\include{listings.inc}
\include{preamble.inc}
\include{macros.inc}
% Полезные макросы листингов.


\title{Coursework}
 

\begin{document}


%Изменим нумерацию на более привычную...
%... и нарушим этим гост.

\renewcommand{\labelenumi}{\arabic{enumi})}
\renewcommand{\labelenumii}{\asbuk{enumii})}


\frontmatter  


\pagestyle{empty} % нумерация выкл.

\begin{center}
    \textup{\textbf{NATIONAL RESEARCH UNIVERSITY} \\ \textbf{HIGHER SCHOOL OF ECONOMICS}} \\[5mm]

     \textup{Faculty of Computer Science  \\ Bachelor’s Programme “Data Science and Business Analytics”} \\[2mm]

 
       \textup{\large\bfseries
         \\[1mm] Research Project Report }\\[5mm] on the topic \\[5mm]
         \textbf{Long-range Camera Guiging for the Person Recognition in Public Spaces.}

\end{center}
\vspace{10mm}
\textbf{Fulfilled by the Student:}\\[2mm]

\begin{tabular}{l@{\hskip 1.5cm}l}
{\hskip -1.5cm}Student of the group \foreignlanguage{russian}{БПАД}212 & \\
{\hskip -1.5cm}Khaibrakhmanov Timur Radikovich\\
{\hskip -1.5cm}\includegraphics[width=0.2\linewidth]{figures/Khaibrakhmanov-IMG/sign.jpg} &Jun 2 2024\\
{\hskip -1.5cm}\rule{4cm}{0.15mm} 

&\rule{3cm}{0.15mm} \vspace{-2mm}\\
\tiny{(signature)} & {\hskip 1.2cm}\tiny{(date)} \\
\end{tabular} \\
\vspace{5mm}\\

{\hskip -1.5cm}\textbf{Checked by the Project Supervisor:}\\[2mm]
\begin{tabular}{l@{\hskip 3cm}l}
Khelvas Alexander Valerievich\\
COO {JSC COS\&HT} \vspace{1mm}\\
 &\\
\rule{4cm}{0.15mm} 

 &\rule{4cm}{0.15mm} \vspace{-2mm}\\
{\hskip 1.5cm}\tiny{(signature)} & {\hskip 1.5cm}\tiny{(date)} \\
\end{tabular}

\vspace{\fill}

\begin{center}
Moscow 2024
\end{center}

%\end{titlepage}

% 

\newpage

\textbf{Список исполнителей}
\\[20mm]


  \begin{tabular}{p{150pt}p{100pt}p{100pt}} 
       Руководитель работы & &~ \\ [15pt]
       Исполнитель & &  \\ [25pt]
         & &~ \\ [15pt]

     
       
        
        
    \end{tabular}

\pagestyle{plain}  

%% Также можно использовать \Referat, как в оригинале
\begin{abstract}

{\color{red}Общие правила 

1) не используем принудительную разметку страниц 

2) картинки лежат в папках figures и eps (при этом все графики и схемы должны быть в eps)  Шрифт на картинках не должен быть меньше 12. Если на картинке используется текст - должны быть русский и английский варианты. 

3) Все публикации на которые ссылаемся должны быть в bibtex в bib файле

\color{blue} 4) Синим цветом помечены куски которые должны быть переизложены для антиплагиата. Это прямые заимствования}




Авторы  

Хайбрахманов Тимур Радикович (trkhaybrakhmanov@edu.hse.ru)

Тимченко Даниил Геннадьевич (dgtimchenko@edu.hse.ru)

Authors  

Khaibrakhmanov Timur Radikovich

Timchenko Daniil Gennadyevich



Объем отчета - 18 стр.%,  таблиц -1 , приложений - 1.

Ключевые слова:  Алгоритм управления длиннофокусной камерой, Задача коммивояжёра (Travelling Salesman Problem, TSP), Распознавание объектов.
 



\end{abstract}
\pagestyle{plain} % нумерация вкл.
  - включить 

\tableofcontents



% %\Defines % Необходимые определения. Вряд ли понадобться

% \chapter{Термины и определения}

% \vspace{20pt}

% \begin{description}

% \item [Искусственный интеллект (далее - ИИ)] - комплекс технологических решений, позволяющий имитировать когнитивные функции человека (включая самообучение и поиск решений без заранее заданного алгоритма) и получать при выполнении конкретных задач результаты, сопоставимые, как минимум, с результатами интеллектуальной деятельности человека. Комплекс технологических решений включает в себя информационно-коммуникационную инфраструктуру, программное обеспечение (в том числе в котором используются методы машинного обучения), процессы и сервисы по обработке данных и поиску решений;
%  \item [g43]
% \end{description}

 
% \Abbreviations %% Список обозначений и сокращений в тексте
% \begin{description}
% \item [ASR] automatic speech recognition (автомаическое распознавание речи)
% \item [CSV]  
%  (CSV от англ. Comma-Separated Values — значения, разделённые запятыми) — текстовый формат, предназначенный для представления табличных данных
% \item [Dynamic HTML]
% набор средств, которые позволяют создавать более интерактивные Web-страницы без увеличения загрузки сервера
% \item [HTML] 
% Язык гипертекстовой разметки документов (от англ. Hypertext Markup Language – “язык гипертекстовой разметки”)
% \item [HTTP] 
% Протокол прикладного уровня для передачи данных, используемый в Web (от англ. HyperText Transfer Protocol - «протокол передачи гипертекста») 
% \item [IP-адрес] 
% Уникальный сетевой адрес узла в компьютерной сети, построенной по протоколу IP
% \item [JavaScript]  Прототипно-ориентированный сценарный язык программирования. Наиболее широкое применение находит в браузерах как язык сценариев для придания интерактивности веб-страницам
% \item [JPEG (JPG)] JPEG - один из популярных графических форматов, применяемый для хранения фотоизображений и подобных им изображений. Файлы, содержащие данные JPEG, обычно имеют расширения .jpg, .jfif, .jpe или .jpeg.
% \item [MS SQL]  Microsoft SQL Server — система управления реляционными базами данных (РСУБД), разработанная корпорацией Microsoft
% \item [PDF]  Portable Document Format (PDF) — межплатформенный формат электронных документов, разработанный фирмой Adobe Systems
% \item [PHP] Cкриптовый язык общего назначения, интенсивно применяемый для разработки веб-приложений.
% \item [PNG]  Растровый формат хранения графической информации, использующий сжатие без потерь качества

% \item [НСИ] 
% Нормативно – справочная информация
% \item [НИР] Научно - исследовательская работа
% \item [АС]  Автоматизированная система
% \item [Интернет]  Информационно-телекоммуникационная сеть Интернет


% \item [Открытые данные] 
% Информация, размещаемая ее обладателями в сети «Интернет» в формате, допускающем автоматизированную обработку без предварительных изменений человеком в целях повторного ее использования

% \item [ПО]  
% Программное обеспечение
% \item[АИС] Автоматизированная информационная система. Но надо протестировать длинные строки в определениях.
% \item[АРМ] Автоматизированная рабочее место
% \item[КВиВ] Подсистема комплексного мониторинга компонент видеомониторинга и видеоанализа 
% \item[МВЯ] Модуль вычислительного ядра  
% \item[МВЯ] Модуль хранения видеоинформации (архив) 
% \item[CPU] Central processing unit 
% \end{description}

% %%% Local Variables:
% %%% mode: latex
% %%% TeX-master: "rpz"
% %%% End:



%\include{12-intro}


\mainmatter 


% \include{1-Osnovania}  % Постановка задачи

\chapter*{Research}

\paragraph{Objective of the research} - development of algorithms for controlling a long-focus camera observing many objects moving in two-dimensional or three-dimensional space.


\paragraph{Goals of the study} 

\begin{itemize}
    \item Analysis of available scientific and patent literature on the topic.
    \item Preparation and marking of data used in the project.
    \item Modeling and its software implementation.
    \item Analysis of results and preparation of publication based on the results of the work.
\end{itemize}


\paragraph{Planned results} 
\begin{itemize}
    \item Labeled data array.
    \item Software-implemented model.
    \item Scientific and technical report.
    \item Publication in "Computer Research and Modeling".
    
\end{itemize}

\textbf{Keywords}: Long-focus camera control algorithm, Traveling Salesman Problem (TSP), Object recognition.
\vspace{2cm}

All code and data for project are available in \href{https://github.com/mikecarti/hse-cam2023}{github repository}

\chapter{Description of the problem setting for the course work}


\section{General Work Plan}

\begin{enumerate}
    \item Delve into the research topic and get familiar with the associated literature.
    \item Solve the TSP problem for navigating players statically positioned on the field.
    \item Develop a simulation for debugging and testing the long-focus camera control algorithm.
    \begin{enumerate}
        \item Process pre-annotated data to build a heat map of player movement.
        \item Based on heat maps, implement a model of player movement on the field, closely matching their real-game movements.
        \item Develop a metric to fairly evaluate the simulation performance.
        \item Integrate agent behavior simulation to camera simulation.
    \end{enumerate}
    \item Evaluate the results and discuss them, derive conclusions.
\end{enumerate}

\begin{figure}[!ht]
    \centering
    \includesvg[width=1.0\textwidth]{figures/Khaibrakhmanov-IMG/Gen_pipeline.svg}
    \caption{General Pipeline}
    \label{fig:gen-pipe}
\end{figure}

The general pipeline of the whole project (in association with Timchenko Daniil) is presented on the Figure~\ref{fig:gen-pipe}. Personally, I was assigned to develop the solution to the TSP problem for navigating players statically positioned on the field, as well as the implementation of a model of player movement part. 

\section{Problem Statement}
\subsection{Static TSP}
Given a football field on which there are players and a ball, the task is to provide an ordered list of player id's (player numbers) for a particular timestamp, which is a frame in our case.

\subsection{Simulation}
Given two datasets: one for the 'Home' team and another for the 'Away' team — each containing information about the positions of the players and the ball the task is to create a model that closely resembles real-world soccer game data for both teams. To evaluate the simulation, we’ll use the following metrics:

\begin{itemize}
    \item Average mean speed of a player
    \item Average variance of speed of a player
    \item Mean average directional vector
\end{itemize}

\subsection{Approximations and limitations}
\begin{itemize}
    \item If one has the intentions to have many more amenities than football players, as well as a distinct quantity of facilities, then another algorithmic approach should be considered.
    \item The camera should be brought back to the center of the field, as it starts there.
\end{itemize}





\chapter{Formal problem statement}
\section{Static TSP}

Given a weighted graph $G = (V, E)$, in our case complete, since from any vertex it can be rational to move to another, in the general case, where

$V$ - number of objects recognized in the frame, at this stage football players, $V = {1, 2, 3, ..., k}$

$E$ - edges of this graph;

$d_{ij}(i, j \in V, i \neq j)$ - the distance between two vertices $i, j$, and $d_{ij} > 0$, $d_{ij} \neq \infty$ and $d_{ij} = d_{ji}$.

We need to find a Hamiltonian cycle (closed path) such that
\begin{align}
     minD = \sum_{i,j \in V} d_{ij} * x_{ij},
\end{align}
Where
\begin{equation}
     x_{ij} =
     \begin{cases}
         1 & e_{ij} \in \text{optimal path}\\
         0 & e_{ij} \notin \text{optimal path}
     \end{cases}
\end{equation}
That is, $x_{ij}$ is a logical variable that turns to 1 if the edge $e_{ij}$ satisfies the condition of belonging to the optimal path, and 0 otherwise. The start and end of the round generally take place in the center of the football field. The result of the algorithm should be a path containing all the vertices of the graph and satisfying the conditions specified above.

\section{Simulation}
Given a football field on which there are players and a ball, specified by coordinates $\vec Y$, as functions of time $i$, in the reference system associated with the field. So we have an input array $Y^{nm}_i$, where
\begin{itemize}
    \item $n$ - player number;
    \item $m$ - describes one of the coordinates of the player’s position;
    \item $i$ - describes a timestamp.
\end{itemize}
Mean speed for a player is calculated as: 
\begin{align}
    \bar v = \frac{1}{T}\sum_{t = 1}^{T}\sqrt{(x_t - x_{t+1})^2 + (y_t - y_{t+1})^2}
\end{align}, where 

\begin{align}
    v = \sqrt{(x_t - x_{t+1})^2 + (y_t - y_{t+1})^2}
\end{align}
is a speed of a player for the timestamps $t$ and $t+1$. Variance of speed for a player is calculated as:
\begin{align}
    \sigma^2 = \frac{1}{T-1}\sum_{t = 1}^{T}(v_t - \bar v)^2
\end{align}
Then, the average direction vector is formed as:
\begin{align}
    \vec D = \frac{1}{T}\sum_{t=1}^{T}\frac{\vec v_t}{||\vec v_t||}
\end{align}, where 
\begin{align}
    \vec v_t = <x_t - x_{t+1}, y_t - y_{t+1}>
\end{align}
After that, an average mean speed, average variance of speed and the mean average direction vectors can be acquired for all players as of the following:
\begin{align}
    V_A= \frac{1}{N}\sum_{n=1}^{N}\bar v_n \\
    \sigma_A^2 = \frac{1}{N}\sum_{n=1}^{N} \sigma_n^2 \\
    \vec D_A = \frac{1}{N}\sum_{n=1}^{N} \vec D_n
\end{align} where A is either "simulation" or "data" and N is the number of players, which is reduced to 20, as the goalkeepers for each team are excluded, since their behavior is drastically different from the remaining field players and the mean speed and the variance of speed would be significantly smaller than those of the other players, which could have a severe impact further on. Thus, the objective functions to be minimised are:
\begin{align}
    \begin{cases}
    min{|V_{simulation} - V_{data}|} \\
    min{|\sigma_{simulation}^2 - \sigma_{data}^2|} \\
    min{|\vec D_{simulation} - \vec D_{data}|}
    \end{cases}
\end{align}
      
 
\chapter{Solution}

\section{Static TSP}
\begin{figure}[ht!]
    \centering
    \includesvg[width=1.0\textwidth]{figures/Khaibrakhmanov-IMG/Pipeline_Static.svg}
    \caption{Static TSP Pipeline}
    \label{fig:tsp}
\end{figure}
The Static TSP solution pipeline is presented on the figure~\ref{fig:tsp}.

There are many solutions to the TSP problem, but there are several key differences between them, the main one being the accuracy. Some algorithms can solve the problem of finding the minimal closed path exactly, while others do not guarantee the minimal path but are significantly faster. For this specific task, the exact solution option was chosen since the number $n$, i.e., the number of recognized objects in the frame, is rather small. Had it been larger, there would possibly be a need to reconsider a solution algorithm.

Thus, according to the article~\cite{Zhang_2021}, the most optimal option was the dynamic programming method, since among the exact approaches to solving TSP, it showed the highest speed. The implemented algorithm takes a data frame of one frame with recognized objects with already transformed coordinates and returns a path that includes the id's of all recognized objects in the frame in the order that minimizes the Hamiltonian closed path in the graph, whose vertices are the id's, as well as the total path length. 
\newpage
\section{Simulation}
\begin{figure}[ht!]
    \centering
    \includesvg[width=1.0\textwidth]{figures/Khaibrakhmanov-IMG/Sim_pipeline.svg}
    \caption{Static TSP Pipeline}
    \label{fig:sim}
\end{figure}
The simulation solution pipeline is presented on the figure~\ref{fig:sim}.
A higher-level description of the simulation logic is the following:
At each timestamp - every 0.04 seconds a set of actions $A_{alpha}$, $A_{beta}$ for each of the teams $T_{alpha}$, $T_{beta}$ is generated, based on the current positions of all of the players on the field, the ball position, player's sub-class and the team mode ($M$), which could be either {'offensive', 'neutral', 'defensive'}. At each time stamp each individual player has to move in some direction, or if the player is controlling the ball he can also hit it towards the target. An example of the 'DefencePlayer' class logic is demonstrated above in the structured pseudo code:
\lstinputlisting[numbers=none]{listings/defence_player.txt}

For each team all of the players are divided into sub-classes of either {'Goalkeeper', 'Defender', 'Midfielder', 'Forward'}. Each sub-class has its own unique decision-making logic and a special parameter zone - an area, where this player would most-likely spend most of the time. 

Before the game start all of the players are arranged according to one of the well-known formation schemes. After the game starts for the duration of the simulation, which is pre-defined coordinates of the players are written to a data frame, which is later used for the evaluation. 

Some other approaches were also tackled, for instance there were several attempts to implement an approach, discussed in \cite{FONSECA20121652}, introducing the usage of the Voronoi diagrams, which would allow a rather more sophisticated addressing of the spatial distribution patterns. However, this idea did not lead to much success due to high computational costs. Although this concept did not catch on, a valuable insight of players from the team with the ball being further apart from each other, whereas defenders being closer to each other was adopted and transferred into the simulation. An example of such behavior can be seen in the defending players trying to 'cut' the opponents and intercept the passed between the two teammates ball, moving towards the mid point of their closest opposing player and the player with the ball in advance.

Another interesting concept, which could be introduced to the model is an approach presented in \cite{boudine2024new}, a 4-network framework, which is a mathematical model, considering the four forces having impact on a decision of a player with a ball: the time to make a decision, the probability $p_{i, j}(\tau_i)$ that the player $t_j$ will receive a pass from the player $t_i$ in comfortable conditions, the risk $r_j$, which measures  the threatening of the adversary’s goal if the ball reaches the player $t_j$ and a score probability $s_i$ of the i-th player. Unfortunately, this framework would require a severely new concept of the game style of a team, not applicable due to the high computational expenses.

Finally, what could have been done is a physics-based model, such as in \cite{Spearman} or in \cite{Alguacil2019ModellingTC}, which represent the so-called time-to-intercept methods and account various fundamental forces that influence the trajectory of a ball. Those advancements are to be studied and implemented in the current model.

       

\chapter{Results}
\section{Simulation Evaluation}
To evaluate the results 20 simulations, the duration of each being 45 minutes, were generated. For the 20 resulting datasets the evaluation statistics being the mean speed, the speed variance and the average direction vector were calculated and then averaged. Then the evaluation metrics were calculated after conducting the same procedure for the given dataset. The results are presented in the table below:
\begin{table}[!ht]
    \centering
    \begin{tabular}{c|c}
        \hline
        \textbf{Metric} & \textbf{Value} \\
        \hline
        Difference in Mean Speed &  0.04\\
        Difference in Variance of Speed & 2.66 \\
        Difference in average direction on x & 0.11\\
        Difference in average direction on y & 0.71
    \end{tabular}
    \caption{Evaluation}
    \label{tab:sim_eval}
\end{table}

The average direction on x and y is given in radians. Based on the following results, it can be concluded that the difference in mean speed and difference in average direction on x can be considered satisfactory. However poor results on the variance of speed and difference in average direction on y should be examined. The high variance difference is most likely related to the inability to grasp all of the nuances of the player logic, those being the specific decisions, like dribbling, standing in place and a sudden acceleration. High difference between the average direction on y can be explained by the team style and by the randomness overall: since we are lacking the real world data to compare the simulations with, other teams may have acquired various strategies, which might have included different spatial patterns. 




%\include{6-specification}
%\section{Спецификация оборудования}

Перечень оборудования, поставленного по   Муниципальному контракту №0350300011821000371\_175478 от 04.10.2021г. на поставку серверного оборудования для подсистемы видеонаблюдения АПК «Безопасный город», серверного оборудования к существующей инфраструктуре Заказчика приведен в таблице \ref{tab:spec}.

\begin{landscape} 
\begin{center}
\begin{longtable}{|l|l|l|l|}
\caption{Спецификация оборудования поставленного по   Муниципальному контракту №0350300011821000371\_175478 от 04.10.2021г.} \label{tab:spec}  \\

\hline \multicolumn{1}{|c|}{\textbf{№}} & \multicolumn{1}{p{110pt}|}{\textbf{Наименование }} & \multicolumn{1}{p{160pt}|}{\textbf{Спецификация}}  & 
\multicolumn{1}{p{70pt}|}{\textbf{Имя и IP}} \\ \hline 
\endfirsthead

\multicolumn{4}{p{340pt}}%
{{\bfseries \tablename\ \thetable{} -- продолжение}} \\
\hline \multicolumn{1}{|c|}{\textbf{№}} & \multicolumn{1}{p{110pt}|}{\textbf{Наименование }} & \multicolumn{1}{p{160pt}|}{\textbf{Спецификация}} & 
\multicolumn{1}{p{70pt}|}{\textbf{Имя и IP}}  \\ \hline 
\endhead

\hline \multicolumn{4}{|r|}{{Продолжение на сл.странице}} \\ \hline
\endfoot

\hline \hline
\endlastfoot

1 & МВЯ -1 & Intel  i9 10900x, ddr4 32gb, sad 120gb,sfp+ 10gbit, RTX3060ti x2  & MPC01  194.255.255.0 \\\hline
2 & МВЯ -2 & Intel  i9 10900x, ddr4 32gb, sad 120gb,sfp+ 10gbit, RTX3060ti x2    & MPC02 \\\hline
3 & МХВ -1 &Intel   corei7 -9700,ОЗУ 16 Gb , 1хssd 120g+12х6тб, sfp+ 10gbit   & SS01\\\hline
4 & МХВ -2 & Intel   corei7 -9700,ОЗУ 16 Gb , 1хssd 120g+12х6тб, sfp+ 10gbit  & SS02\\\hline
5 & МХВ -3 & Intel   corei7 -9700,ОЗУ 16 Gb , 1хssd 120g+12х6тб, sfp+ 10gbit   & SS03\\\hline
6 & МХВ -4 & Intel   corei7 -9700,ОЗУ 16 Gb , 1хssd 120g+12х6тб, sfp+ 10gbit   & SS04\\\hline
7 & МХВ -5 &Intel   corei7 -9700,ОЗУ 16 Gb , 1хssd 120g+12х6тб, sfp+ 10gbit    & SS05\\\hline
8 & МХВ -6 & Intel   corei7 -9700,ОЗУ 16 Gb , 1хssd 120g+12х6тб, sfp+ 10gbit   & SS06\\\hline
9 & Коммутатор ТШ-2 & MikroTik CRS317  & MTIK317 \\\hline
10 & KVM & Aten & \\\hline
11  & Коммутатор ТШ-1  &MikroTik CRS326   & MTIK326\\\hline
12 & ИБП 10000   & IPPON 10000VA
SNMP card  & \\\hline
13  & ТШ-2   &   & \\\hline
 
\end{longtable}
\end{center}
\end{landscape}  

%\include{7-port-comm}

\backmatter %% Здесь заканчивается нумерованная часть документа и начинаются ссылки и
            %% заключение

%\include{7-conclusion}  % Выводы 

\include{81-biblio}      % Список литературы 

%\include{98-Otzyv}

%\appendix   % Тут идут приложения

%\include{91-appendix2} 

% %\chapter*{Перечень внесенных изменений}
\Large{Лист изменений}
\small
\begin{longtable}{|p{90pt}|p{100pt}|p{260pt}|p{80pt}|}
 
\hline
\textbf{Дата} & \textbf{Автор} & \textbf{Внесенные изменения.}   \\
\hline
\endfirsthead
\multicolumn{4}{c}%
{\tablename\ \thetable\ -- \textit{Продолжение}} \\
\hline
\textbf{Дата} & \textbf{Автор} &\textbf{Внесенные изменения.}   \\
\hline
\endhead
\hline \multicolumn{4}{r}{\textit{Продолжение на следующей странице}} \\
\endfoot
\hline
\endlastfoot
  	 &   &    \\\hline
 	 &   &   \\  \hline
 	 & &    \\\hline
 	 & &    \\\hline
 	 & &    \\\hline 
 	 & &    \\\hline
 	 & &    \\\hline
 	 & &    \\\hline
 	 & &    \\\hline
 	 & &    \\\hline
 	 & &    \\\hline
 	 
\end{longtable}



\end{document}

%%% Local Variables:
%%% mode: latex
%%% TeX-master: t
%%% End:

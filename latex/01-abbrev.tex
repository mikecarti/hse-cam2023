% \Abbreviations %% Список обозначений и сокращений в тексте
% \begin{description}
% \item [ASR] automatic speech recognition (автомаическое распознавание речи)
% \item [CSV]  
%  (CSV от англ. Comma-Separated Values — значения, разделённые запятыми) — текстовый формат, предназначенный для представления табличных данных
% \item [Dynamic HTML]
% набор средств, которые позволяют создавать более интерактивные Web-страницы без увеличения загрузки сервера
% \item [HTML] 
% Язык гипертекстовой разметки документов (от англ. Hypertext Markup Language – “язык гипертекстовой разметки”)
% \item [HTTP] 
% Протокол прикладного уровня для передачи данных, используемый в Web (от англ. HyperText Transfer Protocol - «протокол передачи гипертекста») 
% \item [IP-адрес] 
% Уникальный сетевой адрес узла в компьютерной сети, построенной по протоколу IP
% \item [JavaScript]  Прототипно-ориентированный сценарный язык программирования. Наиболее широкое применение находит в браузерах как язык сценариев для придания интерактивности веб-страницам
% \item [JPEG (JPG)] JPEG - один из популярных графических форматов, применяемый для хранения фотоизображений и подобных им изображений. Файлы, содержащие данные JPEG, обычно имеют расширения .jpg, .jfif, .jpe или .jpeg.
% \item [MS SQL]  Microsoft SQL Server — система управления реляционными базами данных (РСУБД), разработанная корпорацией Microsoft
% \item [PDF]  Portable Document Format (PDF) — межплатформенный формат электронных документов, разработанный фирмой Adobe Systems
% \item [PHP] Cкриптовый язык общего назначения, интенсивно применяемый для разработки веб-приложений.
% \item [PNG]  Растровый формат хранения графической информации, использующий сжатие без потерь качества

% \item [НСИ] 
% Нормативно – справочная информация
% \item [НИР] Научно - исследовательская работа
% \item [АС]  Автоматизированная система
% \item [Интернет]  Информационно-телекоммуникационная сеть Интернет


% \item [Открытые данные] 
% Информация, размещаемая ее обладателями в сети «Интернет» в формате, допускающем автоматизированную обработку без предварительных изменений человеком в целях повторного ее использования

% \item [ПО]  
% Программное обеспечение
% \item[АИС] Автоматизированная информационная система. Но надо протестировать длинные строки в определениях.
% \item[АРМ] Автоматизированная рабочее место
% \item[КВиВ] Подсистема комплексного мониторинга компонент видеомониторинга и видеоанализа 
% \item[МВЯ] Модуль вычислительного ядра  
% \item[МВЯ] Модуль хранения видеоинформации (архив) 
% \item[CPU] Central processing unit 
% \end{description}

% %%% Local Variables:
% %%% mode: latex
% %%% TeX-master: "rpz"
% %%% End:

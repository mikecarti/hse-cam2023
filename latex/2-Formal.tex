\chapter{Формализованная постановка задачи}
\label{cha:Proposal}
  
 \section{Задача обхода множества подвижных точек на поверхности}

{\color{red} Обращаю внимание что это не совсем задача описанная в статье - тут ребра графа меняются взаимосвязанным образом и корректно говорить все таки о задаче в ее первозданном виде.Необходимо тут сформулировать задачу в терминах поиска пути на графе заданного множеством вершин $A$ с длиной ребер вычисляемой, как функция времени. 

И хорошо бы картинками все проиллюстрировать в eps формате}


Пусть $\mathbb{R}^{3}$ - это векторное пространство, а $P_{t}$ = $\{ (x_{1}^{(t)},y_{1}^{(t)}, i_1^{(t)}), \dots,(x_{n}^{(t)},y_{n}^{(t)},i_n^{(t)}) \}$ будет набором наблюдаемых объектов, существующих в этом векторном пространстве, которые расположены на плоскости $z=0$ в момент времени $t \, \in \, \mathbb{N}$. $i \, \in \, \{0,1\}$ определяет, был ли обозрен данный объект в период $t_s < t$ ($x,y \, \in \, \mathbb{R}$). Пусть $\mathcal{P}$
$$\mathcal{P}_{t}(\hat{x}, \hat{y}, \hat{z}, \phi_{t}, \psi_{t}, z_{t},) \to \mathcal{V}$$
будет функцией проекции, которая вычисляет угловые точки проекции поля зрения $\mathcal{V}_{t}$ $\, \in \, \mathbb{R}^{{3\times4}}$ на $z=0$ в момент времени $t$. Здесь $\hat{x}, \hat{y}, \hat{z}$ - координаты камеры, $\phi_{t}, \psi_{t}, \theta_{t}, z_t$,  - тангаж, крен и zoom в момент времени $t$ (угловые координаты поворота).
%% и $\zeta$ - это коэффициент масштабирования.  %%

Пусть $\mathcal{C}$
$$
\mathcal{C}(P_{t}, \mathcal{V}_{t}) \to \begin{bmatrix}
\Delta\phi_{t}  &  \Delta\psi_{t}  &  \Delta z_t
\end{bmatrix}^{T}
$$
будет функцией контроллера, которая принимает решение о управлении направлением камеры и масштабированием для момента времени $t+1$. Вращения камеры затем обновляются следующим образом:
$$
\begin{bmatrix}
\phi_{t+1}   \\
 \psi_{t+1} \\
  z_{t+1} 
\end{bmatrix} = 
\begin{bmatrix}
\phi_{t}   \\
 \psi_{t} \\
  z_{t}
\end{bmatrix} + 
\begin{bmatrix}
\Delta\phi_{t}  \\
 \Delta\psi_{t}  \\
 \Delta z_{t}
\end{bmatrix}
$$

Пусть $\mathcal{I}$ 
$$
\mathcal{I}_{t}(\mathcal{P_{t}}, \mathcal{V}_{t}, p_{1}, p_{2}) \to a \, \in \, \{ 0,1 \}
$$
будет функцией индикатора, определяющей, занимает ли наблюдаемый объект определенную часть пространства на видоискателе и никогда не наблюдался в правильном соотношении ранее ($p_{1} \leq p \leq p2$). $a$ в этом случае является индикатором (Истина или Ложь).

Тогда ограниченная задача оптимизации выглядит следующим образом:
$$
\begin{cases}
\sum\limits_{t=1}^{T} \mathcal{I}_{t}(\mathcal{P_{t}}, \mathcal{V}_{t}, p_{1}, p_{2}) \geq n \\
T \to \min\limits_{\mathcal{C}}
\end{cases}
$$

\section{Альтернативная постановка DTSP в общем случае}
DTSP определен на полном двунаправленном графе $ G = (V, E) $, где $ V $ - это множество вершин размера $ n $, а $ E $ - множество ребер. $ V $ состоит из депо 0 и набора потенциальных агентов. Мы рассматриваем асимметричное расстояние в DTSP. Таким образом, $ E $ включает ребра в обе стороны. Агенты, которых нужно посетить, помещаются в пул агентов $ C $ размером $ c $, где $ C $ является подмножеством $ V $.

Продавец начинает свое путешествие из депо 0 в начале времени (t = 0). Он должен обслужить каждого агента в пуле $ C $ ровно один раз и затем вернуться в депо. Время путешествия от вершины $ i $ к вершине $ j $ зависит от временно-зависимой функции $ g_{ij} (t) $, где $ t $ - это время посещения вершины $ i $. Мы предполагаем, что продавец не ждет на вершине. Это верно, когда соблюдается ограничение FIFO (First-In–First-Out), то есть гарантируется, что если транспортное средство покидает вершину $ i $ для вершины $ j $ в определенное время, любое идентичное транспортное средство, покидающее вершину $ i $ для вершины $ j $ в более позднее время, прибудет позже в вершину $ j $.

Пусть $ x_{ij} $ будет бинарной решающей переменной, которая равна 1, если продавец путешествует от вершины $ i $ к вершине $ j $, и 0 в противном случае.

Пусть $ s_i $ будет временем, когда продавец посещает вершину $ i $. Цель состоит в минимизации общего времени путешествия для посещения всех агентов, то есть

$$
\min_{}\sum_{i \, \in \, \{ 0 \}\cup C}\;\sum_{j \, \in \, \{ 0 \}\cup C} g_{ij}(s_{i})x_{ij}
$$

Набор ограничений:


\begin{align}
\sum_{j \, \in \, \{ 0 \}\cup C} x_{ij} = 1 \quad  \forall i \, \in \, C \\
\sum_{i \, \in \, \{ 0 \}\cup C} x_{ji} =1 \quad  \forall i \, \in \, C \\
s_{0} = 0 \\
s_{i} + g_{ij}(s_{i})x_{ij} = s_{i} + (s_{j}-s_{i})x_{ij} \\
\forall i \, \in \, \{ 0 \} \cup C, j \, \in \, C \\
x_{ij} \, \in \, \{ 0,1 \} 
\end{align}


Ограничения (2) и (3) обеспечивают, что существует только одна входящая и исходящая вершина для агента $ i $. Ограничение (4) является начальным временем комми-вояжера в депо. Ограничения (5) указывают, что время посещения агента $ j $ зависит от времени посещения его предшественника $ i $. Этот набор ограничений также гарантирует, что время посещения на каждой вершине увеличивается вдоль пути (при условии, что $ g_{ij}(t)>0 $). Следовательно, в решении не существует подцикла.

Модель, выраженная формулами (1)–(6), по существу является формулировкой TDTSP. Она является нелинейной из-за временно-зависимой функции $ g_{ij} (t) $. Некоторые исследователи пытаются линеаризовать формулировку, накладывая дополнительные предположения. В отличие от них, в этой формулировке описывается наиболее обобщенная версию. Обратите внимание, что область $ g_{ij}(t) $ является непрерывной. Для удобства сбора данных временное пространство можно дискретизировать в набор $ T $ временных шагов. Таким образом, мы имеем время путешествия от вершины $ i $ к вершине $ j $ вокруг временного шага $ t \in T $ в качестве входных значений, обозначенных как $ d_{ij}(t) $. Здесь мы называем $ [d_{ij}(t)] $ трафиковым паттерном графа $ G $. Затем мы можем приблизить $ d_{ij}(t) $, работая с $ d_{ijt} $.

TDTSP предполагает, что все условия динамики графа известны заранее. На практике, чтобы справиться с динамической средой, мы вводим стохастическую переменную $ \phi_{ij}(t) $ в дополнение к $ g_{ij}(t) $, чтобы решить проблему неопределенности реального времени движения. Тогда фактическое время путешествия от вершины $ i $ к вершине $ j $ в момент времени $ t $, обозначенное как $ f_{ij}(t) $, равно $ f_{ij}(t) = g_{ij}(t) + \phi_{ij}(t) $. 

Чтобы решить другую проблему неопределенности, то есть изменение запросов агентов в динамической среде, мы вводим случайную операцию $ \Omega_{k} $ после того, как камера завершает осмотр $ k $-го агента, обозначенную как

$$
\Omega_{k} = \begin{cases}
1 , & \text{вставить не посещенного агента $ i $ в множество $ C $} \\
0 , & \text{ничего не делать} \\
-1,  & \text{удалить агента $ i $ из множества $ C $}
\end{cases}
$$

DTSP - это задача онлайн-оптимизации. Решить ее эффективно очень сложно. Принимая во внимание проблему масштабирования $ n = 40 $ графа G с инвариантным расположением. Если $ c = 20 $, то количество возможных экземпляров также огромно. Когда учитываются два упомянутых выше динамических аспекта, задача становится еще более сложной. 
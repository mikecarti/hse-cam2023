\chapter{Финансово-экономическое обоснование затрат на создание Аналитической системы ФАНО России} 
\label{cha:appendix1}




Финансово-экономическое обоснование затрат на создание Аналитической системы ФАНО России основано на расчете плановой трудоемкости выполнения работ в сроки, определенные контрактом: выполнение работ в течение месяцев.


\section*{Структура затрат}

Выполнение работ и оказание услуг по созданию Аналитической системы ФАНО России включает в себя четыре этапа:
\begin{itemize}
\item 	Подготовка технического задания на информационную систему;
\item 	Предоставление неисключительных прав на программное обеспечение;
\item 	Установка, настройка и монтаж средств программного обеспечения;
\item 	Проведение предварительных испытаний.
\end{itemize}

Для расчета затрат на создание системы учитывались следующие исходные данные:

\begin{itemize}
\item трудоемкость выполнения работ работниками, непосредственно занятыми при выполнении перечисленных работ по созданию системы;
\item	возможная потребность в материалах и комплектующих;
\item	возможная потребность в уникальном оборудовании (специальном оборудовании) для проведения исследований, испытаний и т.п.;
\item	возможная потребность в других прямых затратах, связанных с выполнением работ;
\item	средний уровень возможных непрямых (накладных, общехозяйственных) затрат, непосредственно не связанных с выполнением работ.
\end{itemize}

\section*{Стоимость работ по 1 этапу "Подготовка технического задания на информационную систему"}

1.	Затраты по статье «Материалы и комплектующие»:	не предусмотрены.

2. Работы сторонними организациями не выполнялись.

3. Затраты по оплате труда в объеме 2 954,10 тыс. рублей связаны с выплатой заработной платы работникам, непосредственно занятым при выполнении работ. Трудоемкость работ по подготовке технического задания на информационную систему рассчитана (оценена) в количестве 43 чел.-мес., исходя из перечня и объема работ в рамках первого этапа работ. В расчете затрат по оплате труда использован среднемесячный уровень заработной платы исполнителей, основной состав которых представляют высококвалифицированные аналитики и программисты, принят в размере 68 700 рублей. Это соответствует уровню оплаты труда указанных специалистов в РФ.

4. Страховые взносы рассчитаны в соответствии со ст.58.2 Федерального закона от 24.07.2009 № 212-ФЗ (ред. от 03.07.2016) «О страховых взносах в Пенсионный фонд Российской Федерации, Фонд социального страхования Российской Федерации Федеральный фонд обязательного медицинского страхования» и ст. 22 Федерального закона от 24.07.1998 №125-ФЗ (ред. от 29.12.2015) «Об обязательном социальном страховании от несчастных случаев на производстве и профессиональных заболеваний» - в размере 30,2\% от расходов на оплату труда работников организации-исполнителя, занятым по трудовому договору и составляют 892,14 тыс. руб.

5.	Расходы по статье «Стоимость спецоборудования и специальной оснастки, предназначенных для использования в качестве объектов испытаний и исследований»: не предусмотрены.

6.	Расходы по статье «Прочие прямые расходы», связанные с выполнением проекта: не предусмотрены.

7.	Общехозяйственные не прямые (накладные) расходы взяты для оценочного расчета в размере не более 15\% от суммы всех прямых расходов, и составляют - 391,05 тыс. руб. Информация об обычном объеме общехозяйственных (непрямых) расходах получена исходя из анализа контрактов на создание и развитие информационных систем, размещенных на официальном сайте http://zakupki.gov.ru в сети Интернет.

Стоимость работ по созданию системы, рассчитанная затратным методом оценки, составляет - 4 237,29 тыс. руб.

НДС не облагается на основании п.2 ст. 346.11 НК РФ. 

Расчет стоимости работ, выполненный на основе исходных данных и результатов экспертных оценок, приведен в Таблице \ref{tab:feo1}:


\begin{center}
  \begin{longtable}{|p{40pt}|p{280pt}|p{90pt}|}
    \caption{ Расчет стоимости работ по 1 этапу "Подготовка технического задания на информационную систему"}
    \label{tab:feo1}
    \\ \hline
    № п/п & Наименование статей затрат& Стоимость, руб. \\
    \hline \endfirsthead
    \subcaption{Продолжение таблицы~\ref{tab:feo1}}
    \\ \hline \endhead
    \hline \subcaption{Продолжение на след. стр.}
    \endfoot
    \hline \endlastfoot
   

1	& Материалы и комплектующие	& 0,00 \\ \hline
2	& Затраты по работам, выполняемым сторонними организациями	& 0,00 \\ \hline
3	& Затраты на заработную плату работникам, непосредственно занятым при выполнении работ (сумма подстрок 3.1 и 3.2), в том числе:	& 0,00 \\ \hline
3.1	& - затраты на заработную плату работникам организации- исполнителя, занятым по трудовому договору	& 0,00 \\ \hline
3.2	& - затраты на заработную плату работникам, занятым по договорам гражданско-правового характера	& 0,00 \\ \hline
4	& Затраты на социальное страхование (сумма подстрок 4.1 и 4.2), в том числе: &	0,00 \\ \hline
4.1	& - затраты на социальное страхование (страховые взносы) на оплату труда работников организации-исполнителя, занятым по трудовому договору	& 0,00 \\ \hline
4.2	& - затраты на социальное страхование (страховые взносы) на оплату труда работникам, занятым по договорам гражданско-правового характера	& 0,00 \\ \hline
5	& Стоимость спецоборудования и специальной оснастки, предназначенных для использования в качестве объектов испытаний и исследований	& 0,00 \\ \hline
6 &	Прочие прямые расходы, непосредственно связанные с выполнением работ (сумма подстрок 6.1 и 6.2), в том числе:	& 0,00 \\ \hline
6.1	& - затраты на командировки	& 0,00 \\ \hline
6.2	& - прочие прямые расходы	& 0,00 \\ \hline
7	& Всего прямых расходов (сумма строк 1, 2, 3, 4, 5, 6) &	0,00 \\ \hline
8	& Общехозяйственные (не прямые) расходы, непосредственно не связанные с выполнением работ (не более 15\% от суммы строк 1, 2, 3, 4, 5, 6)	& 0,00 \\ \hline
9	& Итого (сумма строк 7 и 8)	& 2~099~790,00 \\ \hline
10	& НДС, 18 \%: не облагается & --- \\ \hline
11 & Стоимость с учетом НДС (сумма строк 9 и 10): &	2~099~790,00  \\ \hline

  \end{longtable}
\end{center}

Стоимость 1 этапа работ по созданию Аналитической системы ФАНО России, рассчитанная затратным методом оценки, составляет - 2~099~790,00 (Два миллиона девяносто девять тысяч семьсот девяносто) руб. НДС не облагается.


\section*{Стоимость работ по 2 этапу "Предоставление неисключительных прав на программное обеспечение"}


1.	Затраты по статье «Материалы и комплектующие»:	не предусмотрены.

2. Работы сторонними организациями не выполнялись.

3. Затраты по оплате труда в объеме 2 954,10 тыс. рублей связаны с выплатой заработной платы работникам, непосредственно занятым при выполнении работ. Трудоемкость работ по предоставлению неисключительных прав на программное обеспечение рассчитана (оценена) в количестве 43 чел.-мес., исходя из перечня и объема работ в рамках второго этапа работ. В расчете затрат по оплате труда использован среднемесячный уровень заработной платы исполнителей, основной состав которых представляют высококвалифицированные аналитики и программисты, принят в размере 68 700 рублей. Это соответствует уровню оплаты труда указанных специалистов в РФ.

4. Страховые взносы рассчитаны в соответствии со ст.58.2 Федерального закона от 24.07.2009 № 212-ФЗ (ред. от 03.07.2016) «О страховых взносах в Пенсионный фонд Российской Федерации, Фонд социального страхования Российской Федерации Федеральный фонд обязательного медицинского страхования» и ст. 22 Федерального закона от 24.07.1998 №125-ФЗ (ред. от 29.12.2015)	«Об обязательном социальном страховании от несчастных случаев на производстве и профессиональных заболеваний» - в размере 30,2\% от расходов на оплату труда работников организации-исполнителя, занятым по трудовому договору и составляют 892,14 тыс. руб.

5.	Расходы по статье «Стоимость спецоборудования и специальной оснастки, предназначенных для использования в качестве объектов испытаний и исследований»: не предусмотрены.

6.	Расходы по статье «Прочие прямые расходы», связанные с выполнением проекта: не предусмотрены.

7.	Общехозяйственные не прямые (накладные) расходы взяты для оценочного расчета в размере не более 15\% от суммы всех прямых расходов, и составляют - 391,05 тыс. руб. Информация об обычном объеме общехозяйственных (непрямых) расходах получена исходя из анализа контрактов на создание и развитие информационных систем, размещенных на официальном сайте http://zakupki.gov.ru в сети Интернет.

Стоимость работ по созданию системы, рассчитанная затратным методом оценки, составляет - 4 237,29 тыс. руб.

НДС не облагается на основании п.2 ст. 346.11 НК РФ. 

Расчет стоимости работ, выполненный на основе исходных данных и результатов экспертных оценок, приведен в Таблице \ref{tab:feo2}:


\begin{center}
  \begin{longtable}{|p{40pt}|p{280pt}|p{90pt}|}
    \caption{ Расчет стоимости работ по 2 этапу "Предоставление неисключительных прав на программное обеспечение"}
    \label{tab:feo2}
    \\ \hline
    № п/п & Наименование статей затрат& Стоимость, руб. \\
    \hline \endfirsthead
    \subcaption{Продолжение таблицы~\ref{tab:feo2}}
    \\ \hline \endhead
    \hline \subcaption{Продолжение на след. стр.}
    \endfoot
    \hline \endlastfoot
   

1	& Материалы и комплектующие	& 0,00 \\ \hline
2	& Затраты по работам, выполняемым сторонними организациями	& 0,00 \\ \hline
3	& Затраты на заработную плату работникам, непосредственно занятым при выполнении работ (сумма подстрок 3.1 и 3.2), в том числе:	& 0,00 \\ \hline
3.1	& - затраты на заработную плату работникам организации- исполнителя, занятым по трудовому договору	& 0,00 \\ \hline
3.2	& - затраты на заработную плату работникам, занятым по договорам гражданско-правового характера	& 0,00 \\ \hline
4	& Затраты на социальное страхование (сумма подстрок 4.1 и 4.2), в том числе: &	0,00 \\ \hline
4.1	& - затраты на социальное страхование (страховые взносы) на оплату труда работников организации-исполнителя, занятым по трудовому договору	& 0,00 \\ \hline
4.2	& - затраты на социальное страхование (страховые взносы) на оплату труда работникам, занятым по договорам гражданско-правового характера	& 0,00 \\ \hline
5	& Стоимость спецоборудования и специальной оснастки, предназначенных для использования в качестве объектов испытаний и исследований	& 0,00 \\ \hline
6 &	Прочие прямые расходы, непосредственно связанные с выполнением работ (сумма подстрок 6.1 и 6.2), в том числе:	& 0,00 \\ \hline
6.1	& - затраты на командировки	& 0,00 \\ \hline
6.2	& - прочие прямые расходы	& 0,00 \\ \hline
7	& Всего прямых расходов (сумма строк 1, 2, 3, 4, 5, 6) &	0,00 \\ \hline
8	& Общехозяйственные (не прямые) расходы, непосредственно не связанные с выполнением работ (не более 15\% от суммы строк 1, 2, 3, 4, 5, 6)	& 0,00 \\ \hline
9	& Итого (сумма строк 7 и 8)	& 3~149~685,00 \\ \hline
10	& НДС, 18 \%: не облагается & --- \\ \hline
11 & Стоимость с учетом НДС (сумма строк 9 и 10): &	3~149~685,00 \\ \hline

  \end{longtable}
\end{center}

Стоимость 2 этапа работ по созданию Аналитической системы ФАНО России, рассчитанная затратным методом оценки, составляет - 3~149~685,00 (Три миллиона сто сорок девять тысяч шестьсот восемьдесят пять) руб. НДС не облагается.


\section*{Стоимость работ по 3 этапу "Установка, настройка и монтаж средств программного обеспечения"}

1.	Затраты по статье «Материалы и комплектующие»:	не предусмотрены.

2. Работы сторонними организациями не выполнялись.

3. Затраты по оплате труда в объеме 2 954,10 тыс. рублей связаны с выплатой заработной платы работникам, непосредственно занятым при выполнении работ. Трудоемкость работ по установке, настройке и монтажу средств программного обеспечения рассчитана (оценена) в количестве 43 чел.-мес., исходя из перечня и объема работ в рамках третьего этапа работ. В расчете затрат по оплате труда использован среднемесячный уровень заработной платы исполнителей, основной состав которых представляют высококвалифицированные аналитики и программисты, принят в размере 68 700 рублей. Это соответствует уровню оплаты труда указанных специалистов в РФ.

4. Страховые взносы рассчитаны в соответствии со ст.58.2 Федерального закона от 24.07.2009 № 212-ФЗ (ред. от 03.07.2016) «О страховых взносах в Пенсионный фонд Российской Федерации, Фонд социального страхования Российской Федерации Федеральный фонд обязательного медицинского страхования» и ст. 22 Федерального закона от 24.07.1998 №125-ФЗ (ред. от 29.12.2015)	«Об обязательном социальном страховании от несчастных случаев на производстве и профессиональных заболеваний» - в размере 30,2\% от расходов на оплату труда работников организации-исполнителя, занятым по трудовому договору и составляют 892,14 тыс. руб.

5.	Расходы по статье «Стоимость спецоборудования и специальной оснастки, предназначенных для использования в качестве объектов испытаний и исследований»: не предусмотрены.

6.	Расходы по статье «Прочие прямые расходы», связанные с выполнением проекта: не предусмотрены.

7.	Общехозяйственные не прямые (накладные) расходы взяты для оценочного расчета в размере не более 15\% от суммы всех прямых расходов, и составляют - 391,05 тыс. руб. Информация об обычном объеме общехозяйственных (непрямых) расходах получена исходя из анализа контрактов на создание и развитие информационных систем, размещенных на официальном сайте http://zakupki.gov.ru в сети Интернет.

Стоимость работ по созданию системы, рассчитанная затратным методом оценки, составляет - 4 237,29 тыс. руб.

НДС не облагается на основании п.2 ст. 346.11 НК РФ. 

Расчет стоимости работ, выполненный на основе исходных данных и результатов экспертных оценок, приведен в Таблице \ref{tab:feo3}:


\begin{center}
  \begin{longtable}{|p{40pt}|p{280pt}|p{90pt}|}
    \caption{ Расчет стоимости работ по 3 этапу "Установка, настройка и монтаж средств программного обеспечения"}
    \label{tab:feo3}
    \\ \hline
    № п/п & Наименование статей затрат& Стоимость, руб. \\
    \hline \endfirsthead
    \subcaption{Продолжение таблицы~\ref{tab:feo3}}
    \\ \hline \endhead
    \hline \subcaption{Продолжение на след. стр.}
    \endfoot
    \hline \endlastfoot
   

1	& Материалы и комплектующие	& 0,00 \\ \hline
2	& Затраты по работам, выполняемым сторонними организациями	& 0,00 \\ \hline
3	& Затраты на заработную плату работникам, непосредственно занятым при выполнении работ (сумма подстрок 3.1 и 3.2), в том числе:	& 0,00 \\ \hline
3.1	& - затраты на заработную плату работникам организации- исполнителя, занятым по трудовому договору	& 0,00 \\ \hline
3.2	& - затраты на заработную плату работникам, занятым по договорам гражданско-правового характера	& 0,00 \\ \hline
4	& Затраты на социальное страхование (сумма подстрок 4.1 и 4.2), в том числе: &	0,00 \\ \hline
4.1	& - затраты на социальное страхование (страховые взносы) на оплату труда работников организации-исполнителя, занятым по трудовому договору	& 0,00 \\ \hline
4.2	& - затраты на социальное страхование (страховые взносы) на оплату труда работникам, занятым по договорам гражданско-правового характера	& 0,00 \\ \hline
5	& Стоимость спецоборудования и специальной оснастки, предназначенных для использования в качестве объектов испытаний и исследований	& 0,00 \\ \hline
6 &	Прочие прямые расходы, непосредственно связанные с выполнением работ (сумма подстрок 6.1 и 6.2), в том числе:	& 0,00 \\ \hline
6.1	& - затраты на командировки	& 0,00 \\ \hline
6.2	& - прочие прямые расходы	& 0,00 \\ \hline
7	& Всего прямых расходов (сумма строк 1, 2, 3, 4, 5, 6) &	0,00 \\ \hline
8	& Общехозяйственные (не прямые) расходы, непосредственно не связанные с выполнением работ (не более 15\% от суммы строк 1, 2, 3, 4, 5, 6)	& 0,00 \\ \hline
9	& Итого (сумма строк 7 и 8)	& 1~049~895,00 \\ \hline
10	& НДС, 18 \%: не облагается & --- \\ \hline
11 & Стоимость с учетом НДС (сумма строк 9 и 10): &	1~049~895,00 \\ \hline

  \end{longtable}
\end{center}

Стоимость 3 этапа работ по созданию Аналитической системы ФАНО России, рассчитанная затратным методом оценки, составляет - 1~049~895,00 (Один миллион сорок девять тысяч восемьсот девяносто пять) руб. НДС не облагается.



\section*{Стоимость работ по 4 этапу "Проведение предварительных испытаний"}


1.	Затраты по статье «Материалы и комплектующие»:	не предусмотрены.

2. Работы сторонними организациями не выполнялись.

3. Затраты по оплате труда в объеме 2 954,10 тыс. рублей связаны с выплатой заработной платы работникам, непосредственно занятым при выполнении работ. Трудоемкость работ по проведению предварительных испытаний рассчитана (оценена) в количестве 43 чел.-мес., исходя из перечня и объема работ в рамках четвертого этапа работ. В расчете затрат по оплате труда использован среднемесячный уровень заработной платы исполнителей, основной состав которых представляют высококвалифицированные аналитики и программисты, принят в размере 68 700 рублей. Это соответствует уровню оплаты труда указанных специалистов в РФ.

4. Страховые взносы рассчитаны в соответствии со ст.58.2 Федерального закона от 24.07.2009 № 212-ФЗ (ред. от 03.07.2016) «О страховых взносах в Пенсионный фонд Российской Федерации, Фонд социального страхования Российской Федерации Федеральный фонд обязательного медицинского страхования» и ст. 22 Федерального закона от 24.07.1998 №125-ФЗ (ред. от 29.12.2015)	«Об обязательном социальном страховании от несчастных случаев на производстве и профессиональных заболеваний» - в размере 30,2\% от расходов на оплату труда работников организации-исполнителя, занятым по трудовому договору и составляют 892,14 тыс. руб.

5.	Расходы по статье «Стоимость спецоборудования и специальной оснастки, предназначенных для использования в качестве объектов испытаний и исследований»: не предусмотрены.

6.	Расходы по статье «Прочие прямые расходы», связанные с выполнением проекта: не предусмотрены.

7.	Общехозяйственные не прямые (накладные) расходы взяты для оценочного расчета в размере не более 15\% от суммы всех прямых расходов, и составляют - 391,05 тыс. руб. Информация об обычном объеме общехозяйственных (непрямых) расходах получена исходя из анализа контрактов на создание и развитие информационных систем, размещенных на официальном сайте http://zakupki.gov.ru в сети Интернет.

Стоимость работ по созданию системы, рассчитанная затратным методом оценки, составляет - 4 237,29 тыс. руб.

НДС не облагается на основании п.2 ст. 346.11 НК РФ. 

Расчет стоимости работ, выполненный на основе исходных данных и результатов экспертных оценок, приведен в Таблице \ref{tab:feo4}:


\begin{center}
  \begin{longtable}{|p{40pt}|p{280pt}|p{90pt}|}
    \caption{ Расчет стоимости работ по 4 этапу "Проведение предварительных испытаний"}
    \label{tab:feo4}
    \\ \hline
    № п/п & Наименование статей затрат& Стоимость, руб. \\
    \hline \endfirsthead
    \subcaption{Продолжение таблицы~\ref{tab:feo4}}
    \\ \hline \endhead
    \hline \subcaption{Продолжение на след. стр.}
    \endfoot
    \hline \endlastfoot
   

1	& Материалы и комплектующие	& 0,00 \\ \hline
2	& Затраты по работам, выполняемым сторонними организациями	& 0,00 \\ \hline
3	& Затраты на заработную плату работникам, непосредственно занятым при выполнении работ (сумма подстрок 3.1 и 3.2), в том числе:	& 0,00 \\ \hline
3.1	& - затраты на заработную плату работникам организации- исполнителя, занятым по трудовому договору	& 0,00 \\ \hline
3.2	& - затраты на заработную плату работникам, занятым по договорам гражданско-правового характера	& 0,00 \\ \hline
4	& Затраты на социальное страхование (сумма подстрок 4.1 и 4.2), в том числе: &	0,00 \\ \hline
4.1	& - затраты на социальное страхование (страховые взносы) на оплату труда работников организации-исполнителя, занятым по трудовому договору	& 0,00 \\ \hline
4.2	& - затраты на социальное страхование (страховые взносы) на оплату труда работникам, занятым по договорам гражданско-правового характера	& 0,00 \\ \hline
5	& Стоимость спецоборудования и специальной оснастки, предназначенных для использования в качестве объектов испытаний и исследований	& 0,00 \\ \hline
6 &	Прочие прямые расходы, непосредственно связанные с выполнением работ (сумма подстрок 6.1 и 6.2), в том числе:	& 0,00 \\ \hline
6.1	& - затраты на командировки	& 0,00 \\ \hline
6.2	& - прочие прямые расходы	& 0,00 \\ \hline
7	& Всего прямых расходов (сумма строк 1, 2, 3, 4, 5, 6) &	0,00 \\ \hline
8	& Общехозяйственные (не прямые) расходы, непосредственно не связанные с выполнением работ (не более 15\% от суммы строк 1, 2, 3, 4, 5, 6)	& 0,00 \\ \hline
9	& Итого (сумма строк 7 и 8)	& 699~930,00\\ \hline
10	& НДС, 18 \%: не облагается & --- \\ \hline
11 & Стоимость с учетом НДС (сумма строк 9 и 10): &	699~930,00 \\ \hline

  \end{longtable}
\end{center}

Стоимость 4 этапа работ по созданию Аналитической системы ФАНО России, рассчитанная затратным методом оценки, составляет - 699~930,00 (Шестьсот  девяносто девять тысяч девятьсот тридцать) руб. НДС не облагается.


\section*{Общая стоимость работ по контракту}

Общая стоимость работ по контракту складывается из совокупной стоимости четырех этапов работ. Соответствующий расчет приведен в таблице   \ref{tab:feo5}:

\begin{center}
  \begin{longtable}{|p{40pt}|p{280pt}|p{90pt}|}
    \caption{ Расчет общей стоимости работ по контракту}
    \label{tab:feo5}
    \\ \hline
    № п/п & Наименование этапа работ& Стоимость, руб. \\
    \hline \endfirsthead
    \subcaption{Продолжение таблицы~\ref{tab:feo5}}
    \\ \hline \endhead
    \hline \subcaption{Продолжение на след. стр.}
    \endfoot
    \hline \endlastfoot
   

1	& Этап 1 	& 2~099~790,00      \\ \hline
2	& Этап 2	& 3~149~685,00  \\ \hline
3	& Этап 3	& 1~049~895,00  \\ \hline
4	& Этап 4    & 699~930,00  \\ \hline
5 & Общая стоимость работ с учетом НДС (сумма строк 1, 2, 3, 4) &	6~999~300,00 \\ \hline

  \end{longtable}
\end{center}


Таким образом, общая стоимость работ по созданию Аналитической системы ФАНО России, рассчитанная затратным методом оценки, составляет - 6 999 300,00 (Шесть миллионов девятьсот девяносто девять тысяч триста) руб. НДС не облагается.
